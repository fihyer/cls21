\chapter{直线与圆的方程}
\label{ch:直线与圆的方程}

\begin{introduction}
  \item 克莱姆法则~\ref{thm:cramer}
  \item 两点距离公式~\ref{def:point-distance}
  \item 点到直线的距离~\ref{def:point-line-distance}
  \item 平行线间的距离~\ref{def:line-line-distance}
\end{introduction}


\section{直线的交点坐标与距离公式}

\subsection{两条直线的交点坐标}

\begin{table}[htbp]
  \caption{两直线的交点坐标\label{tab:color thm}}
  \centering
  \begin{tabular}{ll}
  \toprule
              集合元素及关系
              & 代数表示\\
  \midrule
              点$A$
              & $A(m, n)$ \\
              直线$l$
              & $l: Ax+By+C=0$ \\
              点$A$在直线$l$上
              & $Am+Bn+C=0$\\
            %   \multirow{2}{*}{直线$l_1$与直线$l_2$的交点是$A$}
              直线$l_1$与直线$l_2$的交点是$A$
              &\parbox{8cm}{\begin{equation*}\mbox{方程组} \left \{ \begin{array}{l} A_{1}x+B_{1}y+C_{1} = 0\\ A_{2}x+B_{2}y+C_{2} = 0 \end{array} \right , \mbox{的解是} \left \{ \begin{array}{l} x = m\\ y = n  \end{array} \right . \end{equation*}}\\
  \bottomrule
  \end{tabular}
\end{table}


\begin{theorem}{克莱姆法则}{cramer}
    已知直线$l_1$与$l_2$两条直线的方程组成的方程组如下:
    \begin{eqnarray}\label{eq:liner_equation}
        \left \{
        \begin{array}{l}
            A_1 x + B_1 y = C_1\\
            A_2 x + B_2 y = C_2
        \end{array}	
        \right .
    \end{eqnarray}
    我们定义
        \begin{equation*}
            \Delta =
            \begin{vmatrix}
                A_1&B_1\\
                A_2&B_2
            \end{vmatrix}
            = A_{1}B_{2}-B_{1}A_{2}
        \end{equation*}
        \begin{equation*}
           \Delta_{x}=
           \begin{vmatrix}
               C_1&B_1\\
               C_2&B_2
           \end{vmatrix}
            = C_{1}B_{2}-B_{1}C_{2}
        \end{equation*}
        \begin{equation*}
           \Delta_{y}=
           \begin{vmatrix}
               A_1&C_1\\
               A_2&C_2
           \end{vmatrix}
            = A_{1}C_{2}-C_{1}A_{2}
        \end{equation*}
    那么二元一次方程组\ref{eq:liner_equation}的解为:
        \begin{eqnarray*}
            \left \{
            \begin{array}{l}
                \displaystyle x = \frac{\Delta_{x}}{\Delta}\\
                \\
                \displaystyle y = \frac{\Delta_{y}}{\Delta}
            \end{array}	
            \right .
        \end{eqnarray*}\\
\end{theorem}


\begin{table}[htbp]
  \caption{两直线的位置关系\label{tab:color thm}}
  \centering
  \begin{tabular}{cccc}
  \toprule
              方程组的解
              & 一组
              & 无数组
              & 无解\\
  \midrule
              直线$l_1$与$l_2$的公共点的个数
              & 一个
              & 无数个
              & 零个 \\
              直线$l_1$与$l_2$的位置关系
              & 相交
              & 重合
              & 平行 \\
  \bottomrule
  \end{tabular}
\end{table}


\begin{note}
   两条直线相交的判定方法:
   \begin{enumerate}
       \item 联立直线方程解方程组,若有一个解,则直线相交;
       \item 两条直线斜率都存在且斜率不相等,则直线相交;
       \item 两条直线的斜率一个存在,另一个不存在,则直线相交。
   \end{enumerate} 
\end{note}

\begin{note}
   过两条直线交点的直线方程的求法:
   \begin{enumerate}
       \item 方程组解法:先解方程组求出已知两直线的交点坐标,再结合其他条件写出直线方程。
       \item 直线待定系数法:先假设过两条直线交点的直线方程,再结合条件利用待定系数法求出参数。
   \end{enumerate} 
\end{note}

\subsection{两点间的距离公式}

\begin{definition}{两点间距离公式}{point-distance}
平面上的两点$A(x_a, y_z)$, $B(x_b, y_b)$间的距离公式为:
\begin{equation}
    \abs{AB} = \sqrt{(x_b - x_a)^2 + (y_b - y_a)^2}
\end{equation}
特别的,平面上任意一点$P(x, y)$与原点$O(0,0)$间的距离为:
\begin{equation}
    \abs{OP} = \sqrt{x^2+y^2}
\end{equation}
\end{definition}

\begin{note}
利用坐标法解决平面几何问题:
\begin{enumerate}
    \item 建立(平面直角)坐标系,尽可能将有关元素放在坐标轴上;
    \item 用坐标表示有关的量;
    \item 将几何关系转化为坐标运算;
    \item 把代数运算结果转化为几何结论。
\end{enumerate}
\end{note}

\subsection{点到直线的距离、两条平行线间的距离}

\begin{definition}{点到直线的距离}{point-line-distance}
平面上任意一点$P$到直线$l$的距离,就是点$P$到直线$l$的垂线段的长度。
\end{definition}

\begin{definition}{平行线间的距离}{line-line-distance}
两条平行线健的距离是夹在两平行线健公垂线的长度。
\end{definition}