\chapter{直线与圆的方程}
\label{ch:直线与圆的方程}

\begin{introduction}
  \item 两点距离公式~\ref{def:point_dst}
\end{introduction}


\section{直线的交点坐标与距离公式}

\subsection{两条直线的交点坐标}

\begin{table}[htbp]
  \caption{两直线的交点坐标\label{tab:color thm}}
  \centering
  \begin{tabular}{cc}
  \toprule
              集合元素及关系
              & 代数表示\\
  \midrule
              点$A$
              & $A(m, n)$ \\
              直线$l$
              & $l: Ax+By+C=0$ \\
              点$A$在直线$l$上
              & $Am+Bn+C=0$\\
              直线$l_1$与直线$l_2$的交点是$A$
              & 方程组的解是\\
  \bottomrule
  \end{tabular}
\end{table}


\begin{note}
    二元一次方程组解法:克莱姆法则
    \begin{equation}
    \begin{array}{l}\left{
        A_1 x + B_1 y + C_1 = 0\\
        A_2 x + B_2 y + C_2 = 0
    \end{array}
    \end{equation}
\end{note}


\begin{table}[htbp]
  \caption{两直线的位置关系\label{tab:color thm}}
  \centering
  \begin{tabular}{cccc}
  \toprule
              方程组的解
              & 一组
              & 无数组
              & 无解\\
  \midrule
              直线$l_1$与$l_2$的公共点的个数
              & 一个
              & 无数个
              & 零个 \\
              直线$l_1$与$l_2$的位置关系
              & 相交
              & 重合
              & 平行 \\
  \bottomrule
  \end{tabular}
\end{table}


\begin{note}
    
\end{note}


\subsection{点到直线的距离、两条平行线间的距离}


%第一章题集
\newpage
\begin{problemset}
\end{problemset}