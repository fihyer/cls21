\chapter{直线与圆的方程}
\label{ch:直线与圆的方程}

\begin{introduction}
  \item 克莱姆法则~\ref{thm:cramer}
  \item 两点距离公式~\ref{def:point_dst}
\end{introduction}


\section{直线的交点坐标与距离公式}

\subsection{两条直线的交点坐标}

\begin{table}[htbp]
  \caption{两直线的交点坐标\label{tab:color thm}}
  \centering
  \begin{tabular}{cc}
  \toprule
              集合元素及关系
              & 代数表示\\
  \midrule
              点$A$
              & $A(m, n)$ \\
              直线$l$
              & $l: Ax+By+C=0$ \\
              点$A$在直线$l$上
              & $Am+Bn+C=0$\\
              直线$l_1$与直线$l_2$的交点是$A$
              & 方程组的解是 \begin{eqnarray}
                \left \{
                \begin{array}{l}
                    x = m\\
                    y = n
                \end{array}	
                \right .
                \end{eqnarray}\\
  \bottomrule
  \end{tabular}
\end{table}


\begin{theorem}{克莱姆法则}{cramer}
    已知直线$l_1$与$l_2$两条直线的方程组成的方程组如下:
    \begin{eqnarray}
        \left \{
        \begin{array}{l}
            A_1 x + B_1 y = C_1\\
            A_2 x + B_2 y = C_2
        \end{array}	
        \right .
    \end{eqnarray}
    我们定义
    \begin{enumerate}
        \item  \begin{equation*}
            \Delta =
            \begin{vmatrix}
                A_1&B_1
                A_2&B_2
            \end{vmatrix}
            = A_{1}B_{2}-B_{1}A_{2}
        \end{equation*}
        \item \begin{equation*}
           \Delta_{x}=
           \begin{vmatrix}
               C_1&B_1
               C_2&B_2
           \end{vmatrix}
            = C_{1}B_{2}-B_{1}C_{2}
        \end{equation*}
        \item \begin{equation*}
           Delta_{y}=
           \begin{vmatrix}
               A_1&C_1
               A_2&C_2
           \end{vmatrix}
            = A_{1}C_{2}-C_{1}A_{2}
        \end{equation*}
    \end{enumerate}
    那么上述二元一次方程的解为:
        \begin{eqnarray}
            \left \{
            \begin{array}{l}
                \displaystyle x = \frac{\Delta_{x}}{\Delta}\\
                \displaystyle y = \frac{\Delta_{y}}{\Delta}
            \end{array}	
            \right .
        \end{eqnarray}\\
\end{theorem}


\begin{table}[htbp]
  \caption{两直线的位置关系\label{tab:color thm}}
  \centering
  \begin{tabular}{cccc}
  \toprule
              方程组的解
              & 一组
              & 无数组
              & 无解\\
  \midrule
              直线$l_1$与$l_2$的公共点的个数
              & 一个
              & 无数个
              & 零个 \\
              直线$l_1$与$l_2$的位置关系
              & 相交
              & 重合
              & 平行 \\
  \bottomrule
  \end{tabular}
\end{table}


\begin{note}
   两条直线相交的判定方法:
   \begin{enumerate}
       \item 联立直线方程解方程组,若有一个解,则直线相交;
       \item 两条直线斜率都存在且斜率不相等,则直线相交;
       \item 两条直线的斜率一个存在,另一个不存在,则直线相交。
   \end{enumerate} 
\end{note}


\subsection{点到直线的距离、两条平行线间的距离}


%第一章题集
\newpage
\begin{problemset}
\end{problemset}