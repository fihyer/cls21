\chapter{三角函数}
\label{ch:三角函数}

\begin{introduction}
  \item 任意角的概念~\ref{def:any-angle}
  \item 象限角及轴线角概念~\ref{def:quadrant-angle}
  \item 终边相同的角~\ref{def:angle-set}
  \item 终边相同角推论~\ref{cor:angle-set-col}
  \item 圆心角的弧度~\ref{def:central-angle-radian}
  \item 角度弧度转换
\end{introduction}

\section{任意角和弧度制}

\subsection{任意角}

\begin{remark}{\vspace{0.3mm}角的概念及表示}
\begin{enumerate}
\item 角可以看成平面内\underline{一条射线}绕着它的\underline{端点}旋转所成的图形。
\item 射线的端点叫做角的\underline{顶点}
\item 射线的起始位置叫做角的\underline{始边}
\item 射线的终止位置叫做角的\underline{终边}
\end{enumerate}
%\begin{firute}[h]
\begin{wrapfigure}{r}{0.25\textwidth}
\begin{center}
\begin{tikzpicture}
  \draw
    (1.5,0) coordinate (A) node[right] {A}
    -- (0,0) coordinate (O) node[left] {O}
    -- (-1,1) coordinate (B) node[above left] {B}
    pic["$\alpha$", draw=orange, ->, angle eccentricity=1.2, angle radius=0.4cm]
    {angle=A--O--B};
\end{tikzpicture}	
\end{center}
\end{wrapfigure}
%\end{figure}
如图所示$O$为$\angle{AOB}$的顶点,$OA$为$\angle{AOB}$的始边,$OB$为$\angle{AOB}$的终边。
\end{remark}


\begin{definition}{任意角}{any-angle}
\begin{enumerate}
\item \textbf{正角}:按\textcolor{third}{逆时针}方向旋转形成的角
\item \textbf{负角}:按\textcolor{third}{顺时针}方向旋转形成的角
\item \textbf{零角}:\textcolor{third}{没有作任何旋转}的角
\end{enumerate}
\end{definition}


\begin{note}
设$\alpha$, $\beta$是任意两个角,如果把角$\alpha$的终边旋转角$\beta$,则最终形成的角为角$\alpha + \beta$。
\end{note}


\begin{definition}{象限角和轴线角}{quadrant-angle}
在平面直角坐标系中,将角的顶点与原点重合,角的始边与$x$轴的非负半轴重合:
\begin{enumerate}
\item 终边在坐标轴上的角,不属于任何一个象限,我们称之为\textcolor{third}{轴线角};
\item 终边不在坐标轴上的角我们称之为\textcolor{third}{象限角}。角的终边在第几象限,我们就叫这个角是第几象限角。
\end{enumerate}
\end{definition}


\begin{remark}
\begin{enumerate}
\item 锐角: 大于$0^{\circ}$小于$90^{\circ}$的角。锐角一定是第一象限角,但第一象限角不一定是锐角。
\item 直角: 等于$90^{\circ}$的角,直角是轴线角,其终边在$y$轴的非负半轴。
\item 钝角: 大于$90^{\circ}$小于$180^{\circ}$的角。钝角一定是第二象限角,但第二象限角不一定是钝角,即若存在命题$p$, $q$分别为:
\begin{enumerate}
\item $p$:角 $\alpha $ 是钝角,
\item $q$:角 $\alpha $ 是第二象限角,
\end{enumerate}
则$p \implies q, q \centernot \implies p$,或我们说$p$是$q$的充分不必要条件。
\item 平角: 等于$180^{\circ}$的角,平角是轴线角,其终边在$x$轴的非正半轴。
\item 周角: 等于$360^{\circ}$的角,周角是终边在$x$轴的非负半轴的轴线角。
\end{enumerate}
\end{remark}


\begin{definition}{终边相同的角}{angle-set}
任意角$\alpha$终边相同的角$\beta \in S$
\begin{equation}
S = \{\beta \mid \beta = \alpha + k \cdot 360^{\circ}, k \in \mathbb{Z} \}
\end{equation}
\end{definition}

\begin{corollary}{终边相同的角}{angle-set-col}
\begin{enumerate}
\item 终边相同的角之间相差$360^{\circ}$的整数倍。
\item 终边在同一直线上的角之间相差$180^{\circ}$的整数倍。
\item 终边相互垂直的两条直线上的角之间相差$90^{\circ}$的整数倍。
\end{enumerate}
\end{corollary}

\begin{note}{判断象限角的方法}
\begin{enumerate}
\item 图像法: 根据已知条件在平面直角坐标系中画出相应的角,根据象限角的定义观察终边的位置,确定其为第几象限角。
\item 解析法: 先将角$\alpha$转化为$\alpha = k \cdot 360^{\circ} + \beta \left(k \in \mathbb{Z}, \mbox{且} 0^{\circ} < \beta < 360^{\circ} \right)$的形式;然后判断角$\beta$的终边所在的象限; 角$\beta$是第几象限角,角$\alpha$就是第几象限角。
\end{enumerate}
\end{note}

\begin{note}
\begin{table}[htbp]
  \caption{象限角与轴线角的集合\label{tab:color thm}}
  \centering
  \begin{tabular}{ll}
  \toprule
              分类
              & 集合\\
  \midrule
              终边在$x$轴
              & $\{\alpha \mid \alpha = k \cdot 180^{\circ}, k \in \mathbb{Z}\}$\\
              终边在$x$轴非负半轴
              & $\{\alpha \mid \alpha = k \cdot 360^{\circ}, k \in \mathbb{Z}\}$\\
              终边在$x$轴非正半轴
              & $\{\alpha \mid \alpha = k \cdot 360^{\circ} + 180^{\circ}, k \in \mathbb{Z}\}$ \\
              终边在$y$轴
              & $\{\alpha \mid \alpha = k \cdot 180^{\circ} + 90^{\circ}, k \in \mathbb{Z}\}$\\
              终边在$y$非负半轴
              & $\{\alpha \mid \alpha = k \cdot 360^{\circ} + 90^{\circ}, k \in \mathbb{Z}\}$\\
              终边在$y$非正半轴
              & $\{\alpha \mid \alpha = k \cdot 360^{\circ} + 270^{\circ}, k \in \mathbb{Z}\}$\\
			  终边在$y=x$轴上              
              & $\{\alpha \mid \alpha = k \cdot 180^{\circ} + 45^{\circ}, k \in \mathbb{Z}\}$\\
              终边在坐标轴
              & $\{\alpha \mid \alpha = k \cdot 90^{\circ}, k \in \mathbb{Z}\}$\\
              第一象限角
              & $\{\alpha \mid \alpha = 0^{\circ} + k \cdot 360^{\circ} < \alpha < 90^{\circ} + k \cdot 90^{\circ}, k \in \mathbb{Z}\}$\\
              第二象限角
              & $\{\alpha \mid \alpha = 90^{\circ} + k \cdot 360^{\circ} < \alpha < 180^{\circ} + k \cdot 90^{\circ}, k \in \mathbb{Z}\}$\\
              第三象限角
              & $\{\alpha \mid \alpha = 180^{\circ} + k \cdot 360^{\circ} < \alpha < 270^{\circ} + k \cdot 90^{\circ}, k \in \mathbb{Z}\}$\\
              第四象限角
              & $\{\alpha \mid \alpha = 270^{\circ} + k \cdot 360^{\circ} < \alpha < 360^{\circ} + k \cdot 90^{\circ}, k \in \mathbb{Z}\}$\\
  \bottomrule
  \end{tabular}
\end{table}
\end{note}

\vspace{0.4cm}
\begin{remark}
因为角是由端点和端点出发的两条射线(始边和终边)构成的,因此在直角坐标系上轴线角要么是在$x$轴的非正半轴或非负半轴,要么就是在$y$轴的非正半轴或者非负半轴,而不是$x$的正半轴或负半轴(或$y$轴的正半轴或负半轴)。
\end{remark}

\vspace{0.4cm}

\begin{note}{已知角$\alpha$所在象限,求角$\displaystyle \theta = \frac{\alpha}{n} \mbox{或} \theta = n \alpha \left(n \in {\mathbb{N}}^\star \right) $ 所在象限}
\begin{enumerate}
\item 先将角$\alpha$的范围转化为含$k$的不等式;
\item 两边同乘或者同除$n$;
\item 对$k$进行奇偶性讨论,即
\begin{enumerate}
\item $k = 2n \left( n \in \mathbb{Z} \right)$
\item $k = 2n + 1 \left( n \in \mathbb{Z} \right)$
\end{enumerate} 
时不等式的结果,从而得到$\theta$所在的象限。
\end{enumerate}
\end{note}

\begin{exercise}
写出终边在$y$轴上的角的集合.
\end{exercise}

\begin{solution}
在$0^{\circ} \sim 360^{\circ}$范围内,终边在$y$轴上的角由两个:$90^{\circ} \mbox{和} 270^{\circ}$.\\
因此,所有与$90^{\circ}$
\end{solution}

\subsection{弧度制}

在初中我们知道“度”是将圆平均分成360份,期中的一份所对的角叫做$1$度

\begin{figure}
	\begin{center}
	\begin{tikzpicture}
	\def\R{5cm}
	\draw (0,0) circle[radius=\R];
	\draw (0,0) -- (0:\R) (0,0) -- (1:\R) (0,0) -- (57.30:\R);
	\end{tikzpicture}
	\end{center}
\end{figure}

\begin{definition}{角度制与弧度制}{degree-radian}
\begin{enumerate}
\item 角度:由角(圆心角)所对的\textcolor{third}{圆弧的长度}除以\textcolor{third}{圆的周长}再乘以\textcolor{third}{$360^{\circ}$}为角的角度,一般用“${}^{\circ} $”来标记,读作“度”。一度可以继续分为60“分”或3600“秒”,使用角度为单位度量角的单位制角角度制。

\item 弧度:圆心角所对的\textcolor{third}{圆弧的长度}除以\textcolor{third}{圆的半径}所得的值为角的弧度。一般用“$rad$”来标记弧度(通常可省略不写),读作“弧度”。用弧度作为单位度量角的单位制叫做\textcolor{third}{弧度制}。
\end{enumerate}
\end{definition}
%
%\begin{note}
%若半径为$r$的圆,圆心角$\theta$所对的圆弧长度为$l$,由初中知识我们可知圆的周长$C$为
%\begin{equation}
%C = 2 \pi r
%\end{equation}
%若圆周的360份中的一份所对应的圆心角$\phi$的角度是$1^{\circ}$,即$\displaystyle \abs{\phi}=\frac{1^{\circ}}{360^{\circ}}=\frac{1}{360}$,那么圆周360等分中的$l$份的圆心角$\theta$的角度若为$n^{\circ}$,那么$\displaystyle \theta = n \cdot \phi \mbox{且} \frac{1}{n} = \frac{l}{C}$:
%\begin{equation}
%\abs{\theta} = \frac{l}{C} \cdot 360^{\circ} = \frac{l}{2 \pi r} \cdot 360^{\circ}
%\end{equation}
%若半径为$r=1$的圆心角对应的弧长为$l=r$, 则$\displaystyle 1 \hspace{1mm} rad = \frac{\pi r}{l} \cdot 180^{\circ} $
%\end{note}

\begin{definition}{圆心角的弧度}{central-angle-radian}
在半径为$r$的$\odot O$中,弧长为$l$的弧所对的圆心角为$\alpha rad$则:
\begin{equation}
\displaystyle \abs{\alpha} = \frac{l}{r}
\end{equation}
\end{definition}

\begin{definition}{终边相同的角(弧度制)}{radian-set}
与角$\alpha$终边相同的角$\beta \in S$
\begin{equation}
S = \{\beta \mid \beta = \alpha + 2k\pi, k \in \mathbb{Z}\}
\end{equation}
\end{definition}

\begin{note}
由角度制和弧度制的定义可知
\begin{eqnarray}
180^{\circ} = \pi \hspace{1mm} rad \iff \left \{
\begin{array}{l}
\displaystyle 1^{\circ} = \frac{\pi}{180} \hspace{1mm} rad \approx 0.01745 \hspace{1mm} rad \vspace{0.5cm} \\
\displaystyle 1 \hspace{1mm} rad = {\left( \frac{180}{\pi} \right)}^{\circ} \approx 57.30^{\circ}
\end{array}	
\right .
\end{eqnarray}
\end{note}

\begin{definition}{扇形的弧长和面积公式}{arc-area-def}
若扇形的半径为$R$, 弧长为$l$, 面积为$S$, 且$\alpha \left(0 < \alpha < 2\pi \right)$为扇形的圆心角,则:
\begin{eqnarray}
l=\alpha R \hspace{5mm} \left(\mbox{弧长公式}\right) \\
\displaystyle S=\frac{1}{2}lR=\frac{1}{2}\alpha R^2 \hspace{5mm} \left(\mbox{面积公式}\right)
\end{eqnarray}
\end{definition}

\begin{note}
\begin{enumerate}
\item 零扇形的弧长和面积公式解题时,角的单位必须是弧度。
\item 求扇形面记得最大值时,利用面积公式转化为求二次函数最大值问题。
\item 在解决弧长和扇形面积问题时,要合理利用圆心角所在的三角形。
\end{enumerate}
\end{note}


\section{三角函数的概念}

\subsection{三角函数的概念}

给定任意角$\alpha \in \mathbb{R}$, 它的终边$OP$与单位圆交点$P$的坐标都是角$\alpha$的函数。

\begin{definition}{三角函数}{trig-function}
任意角$\alpha \in \mathbb{R}$, 点$P(x,y)$是角$\alpha$的终边$OP$与单位圆$\odot O$的交点,则:
\begin{enumerate}
\item 正弦函数就是点$P$的纵坐标$y$ ,记作$\sin{\alpha}$,即 $y=\sin{\alpha}$
\item 余弦函数就是点$P$的纵坐标$x$ ,记作$\cos{\alpha}$,即 $x=\cos{\alpha}$
\item 正切就是点$P$的纵坐标$y$与横坐标$x$的比值 ,记作$\tan{\alpha}$,即 $\displaystyle \frac{y}{x}=\tan{\alpha}, (x \neq 0)$
\end{enumerate}
\end{definition}

\begin{conclusion}
$\forall \alpha \in \mathbb{R}$,若$\alpha$终边上任意一点$P(x,y), (x, y) \neq (0, 0)$,则:
\begin{equation}
\begin{array}{l}
\displaystyle \sin{\alpha} = \frac{y}{\sqrt[]{x^2+y^2}} \\
\displaystyle \cos{\alpha} = \frac{x}{\sqrt[]{x^2+y^2}} \\
\displaystyle \tan{\alpha} = \frac{y}{x}
\end{array}
\end{equation}
\end{conclusion}
\subsection{同角三角函数的基本关系}

\begin{definition}{终边相同的角的同一三角函数}{trigo-equ1}
	\begin{equation}
	\begin{array}{l}
		\sin{\alpha} = \sin{\left(\alpha + k \cdot 2 \pi \right)}, k \in \mathbb{Z} \\
		\cos{\alpha} = \cos{\left(\alpha + k \cdot 2 \pi \right)}, k \in \mathbb{Z} \\
		\tan{\alpha} = \tan{\left(\alpha + k \cdot 2 \pi \right)}, k \in \mathbb{Z}	
	\end{array}
	\end{equation}
\end{definition}

利用\textcolor{third}{终边相同的角的同一三角函数值相等}结论,我们可以将求任意角的三角函数值转换到求$\displaystyle 0 \sim 2\pi$(即$0^{\circ} \mid 360^{\circ}$)之间角的三角函数值。

\begin{definition}{同角三角函数之间的关系}{}
${\sin{\alpha}}^2 + {\cos{\alpha}}^2 = 1$
\end{definition}

\section{诱导公式}



\section{三角函数的图像和性质}

\begin{remark}
函数的三要素、三性质
\begin{enumerate}
\item 定义域
\item 对应更关系
\item 值域
\end{enumerate}
\begin{enumerate}
\item 单调性
\item 奇偶性
\item 周期性
\end{enumerate}
\end{remark}

\subsection{正弦函数、余弦函数的图像}
\subsection{正弦函数、余弦函数的性质}
\subsubsection{周期性}
\subsubsection{奇偶性}
\subsubsection{单调性}

\subsection{正切函数的图像}
\subsection{正切函数的性质}
\subsubsection{周期性}
\subsubsection{奇偶性}
\subsubsection{单调性}

\section{三角恒等变换}
\subsection{和角公式}
\subsection{差角公式}
\subsection{倍角公式}

\section{函数$y=A\sin{\omega x + \varphi}$}


\section{三角函数的应用}

\newpage
\begin{problemset}
\item 十九大指出中国的电动汽车革命早已展开,通过以新能源汽车代替汽/柴油车,中国正在大力实施一项将重塑全球汽车行业的计划,2020年某企业计划引进新能源汽车生产设备,通过市场分析,全年需投入固定成本3000万元,每生产$x$(百辆)需要另投入成本$y$(万元),且
\begin{equation*}
y = \left \{
	\begin{array}{l}
		10x^2+100x, 0 < x <40\\
		\displaystyle 501x + \frac{10000}{x}-4500, x \geq 40
  \end{array}
  \right .
\end{equation*}
由市场调研知,每辆车售价$5$万元,且全年内生产的车辆当你那能全部销售完。
\begin{enumerate}
	\item 求出2020年的利润$S$(万元)关于年产量$x$(百辆)的函数关系式;(利润=销售额-成本)
	\item 当2020年产量为多少百辆时,企业所获利润最大?并求出最大利润。
\end{enumerate}

\vspace{4mm}
\item 已知某工厂生产机器设备的年固定成本为200万元。没生产1台还需另投入20万元。设该公司一年内共生产该机器设备$x$台并全部销售完,每台机器设备销售的收入为$R(x)$万元,且
\begin{equation*}
	R(x) = \left \{
	\begin{array}{l}
		\displaystyle 40 + \frac{800}{x}, 0 < x \leq 30\\
		\displaystyle \vspace{2mm} \frac{280\sqrt{x}+1000}{x}, x > 30
  \end{array}
  \right .
\end{equation*}
\begin{enumerate}
	\item 求年利润$y$(万元)关于年产量$x$(台)的函数解析式;
	\item 当年产量为多少台时,该工厂生产所获得的的年利润最大?并求出最大年利润。
\end{enumerate}

\vspace{4mm}
\item 湖南娄底某高科技企业决定开发生产一款大型电子设备。生产这种设备的年固定成本为500万元,每生产$x$台,需要零投入成本$h(x)$(万元),当年生产量小于60台时$h(x)=x^2+20x$(万元);当年产量不少于60台时$\displaystyle h(x)=102x+\frac{9800}{x}-2080$。若每台设备的售价为100万元,通过市场分析,假设该企业生产的电子设备能全部售完。
\begin{enumerate}
	\item 求年利润$L(x)$(万元)关于年产量$x$(台)的函数关系式;
	\item 年产量为多少台时,该企业在这一电子设备的生产中获利最大?
\end{enumerate}
\end{problemset}