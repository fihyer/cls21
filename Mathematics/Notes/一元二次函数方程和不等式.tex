\chapter{一元二次函数、方程和不等式}
\label{ch:一元二次函数方程和不等式}

\begin{introduction}
  \item 不等式的基本性质~\ref{property:inequality}
  \item 不等式的其他性质~\ref{property:inequality-others}
  \item 基本不等式~\ref{def:MA}
  \item 一元二次不等式~\ref{def:polynomial-inequality}
\end{introduction}



\section{等式性质与不等式性质}

\begin{axiom}{基本事实}{}
\begin{enumerate}
\item 如果 $a-b$ 是正数,那么 $a > b$; 即$a>b \iff a-b>0$;
\item 如果 $a-b$ 等于零,那么 $a = b$; 即$a=b \iff a-b=0$;
\item 如果 $a-b$ 是负数,那么 $a < b$; 即$a<b \iff a-b<0$.
\end{enumerate}
\end{axiom}

\begin{note}{比较大小的方法}
\begin{itemize}
        \item 作差比较法的两种情况:
        \begin{itemize}
		\item 将差进行因式分解转化为几个因式相乘;
		\item 将差式通过配方转化为几个非负实数之和,然后判断正负。
        \end{itemize}
        \item 作商比较法通常适用于两代数式同号的情况。
        \item 构造函数:将要比较的两个数作为一个函数的两个函数值,根据函数的单调性得出大小关系。
\end{itemize}
\end{note}

\vspace{1cm}

\begin{property}\label{property:inequality}
不等式的基本性质
\begin{enumerate}
\item 不等式的对称性:如果 $a>b$ 那么 $b<a$ 如果 $b<a$,那么 $a>b$. 即$a>b \iff b<a$.
\item 不等式的传递性:如果 $a>b$, $b>c$, 那么 $a>c$. 即$a>b, b>c \Longrightarrow a>c$.
\item 不等式的可加性:如果 $a>b$, 那么 $a+c>b+c$. 即$a>b \iff a+c>b+c$.
\item 不等式的可乘性:
\begin{enumerate}
\item 如果 $a>b$, $c>0$, 那么 $ac>bc$. 即$a>b, c>0 \Longrightarrow ac>bc$.
\item 如果 $a>b$, $c<0$, 那么 $ac<bc$. 即$a>b, c<0 \Longrightarrow ac<bc$.
\end{enumerate}
\item 不等式的同向可加性:如果 $a>b$, $c>d$, 那么 $a+c>b+d$. 即$a>b, c>d \Longrightarrow a+c>c+d$.
\item 不等式的同向可乘性:如果 $a>b>0$, $c>d>0$, 那么 $a>c$. 即$a>b, b>d \Longrightarrow ac>bd$.
\item 不等式的可乘方性:
\begin{enumerate}
\item 如果 $a>b>0$, 那么 $a^n>b^n$, 即$a>b>0 \Longrightarrow a^n>b^n (n\in N, n \geq 2)$.
\item 如果 $a>b>0 (n \in N, n \geq 2)$, 那么 $\sqrt[n]{b}>\sqrt[n]{b}$. 即$a>b>0 \Longrightarrow \sqrt[n]{a}>\sqrt[n]{b} (n\in N, n \geq 2)$.
\end{enumerate}
\end{enumerate}
\end{property}

\vspace{1cm}
\begin{property}\label{property:inequality-others}
不等式的其他性质
\begin{enumerate}
\item 倒数性质:
  \begin{enumerate}
        \item $\displaystyle a>b , ab>0 \implies \frac{1}{a} < \frac{1}{b}$.
        \item $\displaystyle \frac{1}{a}<\frac{1}{b}, ab>0 \implies a>b$.
        \item $\displaystyle a<0<b \implies \frac{1}{a}<\frac{1}{b}$.
        \item $\displaystyle a>b>0, 0<c<d \implies \frac{a}{c}>\frac{b}{d}$.
        \item $0<a<x<b$ 或 $\displaystyle a<x<b<0 \implies \frac{1}{b}<\frac{1}{x}<\frac{1}{a}$
  \end{enumerate}
\item 分数性质:若 $a>b>0, m>0$ \\
\begin{enumerate}
        \item 则 $\displaystyle \frac{b}{a} < \frac{b+m}{a+m}; \frac{b}{a} > \frac{b-m}{a-m} (b-m>0)$;\\
        \item 则 $\displaystyle \frac{a}{b} > \frac{a+m}{b+m}; \frac{a}{b} < \frac{a-m}{b-m} (b-m>0)$.
\end{enumerate}
\end{enumerate}
\end{property}


\section{基本不等式}

\begin{definition}{基本不等式}{MA}
如果 $a, b \in \mathbb{R}, a>0, b>0$ 那么
\begin{equation}
\displaystyle \frac{a+b}{2} \geq \sqrt{ab} 
\end{equation}
当且仅当 $a=b$ 时, 等号成立。其中, $\displaystyle \frac{a+b}{2}$ 叫做正数 $a, b$ 的算数平均数,$\sqrt{ab}$ 叫做正数 $a, b$ 的几何平均数。
\end{definition}

\begin{note}
\begin{enumerate}
\item 基本不等式中的$a$与$b$都是实数,如果不是实数,例如已知向量的内积$\vec{a}\cdot \vec{b} = 1$, 则$\vec{a} + \vec{b} \geq 2\sqrt[]{\vec{a} \cdot \vec{b}} = 2$,就是错的。
\item 基本不等式中的$a$与$b$这两个符号可以代表数字,也可以代表代数式(单项式、多项式、整式、分式、指数式、对数式、三角式等等),例如:
\begin{eqnarray*}
\displaystyle x + \frac{2}{x} \geq 2\sqrt[]{2}\\
\displaystyle \frac{2}{x} + \frac{x}{2} \geq 2, (x >0)\\
2^x + 2^y \geq 2\sqrt[]{2^(x+y)}\\
\log_{a}{b} + \log_{b}{a} \geq 2 (\log_{a}{b} > 0)\\
\displaystyle \sin{x} + \frac{1}{\sin{x}} \geq 2 (0< \sin{x} \leq 1)\\
\displaystyle \frac{a^2+b^2}{ab} = \frac{a}{b} + \frac{b}{a} \geq 2 (a, b > 0)\\
\end{eqnarray*}
\item 熟悉理解基本不等是的“正向操作”还要注意利用反向操作,即$\displaystyle \sqrt[]{ab} \leq \frac{a+b}{2}$
\end{enumerate}
\end{note}

\begin{note}{重要不等式}
\begin{enumerate}
        \item $a^2+b^2 \geq 2ab (a,b \in \mathbb{R})$, 当且仅当$a=b$ 时取等号;
        \item \vspace{2mm} $\displaystyle ab \leq \left( \frac{a+b}{2}\right)^2 (a, b\in\mathbb{R})$, 当且仅当 $a=b$ 时取等号;
        \item \vspace{2mm} $\displaystyle \sqrt{a^2+b^2} \geq \left( \frac{a+b}{2}\right)^2 (a,b\in\mathbb{R})$, 当且仅当 $a=b$ 时取等号; 
        \item \vspace{2mm} $\displaystyle min(a, b) \leq \frac{2ab}{a+b} \leq \sqrt{ab} \leq \frac{a+b}{2} \leq \sqrt{\frac{a^2+b^2}{2}} \leq max(a, b) $
\end{enumerate}
\end{note}

\begin{remark}
上述“重要不等式笔记”中:
\begin{enumerate}
\item $\displaystyle \frac{2ab}{a+b}=\frac{2}{\frac{1}{a}+\frac{1}{b}}$,被称为“调和平均数”;
\item $\sqrt{ab}$,被称为“几何平均数”
\item $\frac{a+b}{2}$,称为“算数平均数”
\item $\sqrt{\frac{a^2+b^2}{2}}$,称为“平方平均数”
\end{enumerate}

\end{remark}

\begin{theorem}{和定积最大,积定和最小}
设$a$, $b$都是正数:
\begin{enumerate}
\item 若$a+b=S (S\mbox{为常数})$, 那么当$a=b$时,$\displaystyle (ab)_{max}=\frac{S^2}{4}$
\item 若$ab=P (P\mbox{为常数})$, 那么当$a=b$时,$\displaystyle (a+b)_{min}=2\sqrt[]{P}$
\end{enumerate}
\end{theorem}


\begin{note}{基本不等式的性质}
要利用基本不等式求最值首先要满足一下三个条件:
\begin{enumerate}
\item 一正:相加或着想乘的两个数需为正数;
\item 二定:两个数的和或着积为定值(常数值);
\item 三相等:当且仅当相加或者相乘的这两个数相等的时候基本不等式取等号。
\end{enumerate}
\end{note}

\vspace{3mm}
\begin{note}{利用基本不等式求最值的方法}
\begin{enumerate}
\item 直接解法:如果所给条件满足“一正,二定,三相等”,则直接套用基本不等式。
\item 配凑法:根据所给代数式的“外型”配凑除积、和为常数的形式,然后在利用基本不等式求最值。
\item 换元法:
\item 1的代换法:如果所给条件为分数和定值(整数和定值)形式,让求整数和(分数和)的最值,则将所求(和)式子乘以已知式子并乘以定值的倒数,化简后利用基本不等式求最值。
\item 消元法:
\begin{enumerate}
\item 根据所给条件(一般为方程)建立两个变量之间的(代数)函数关系,将其中一个变量代入所求代数式中
\end{enumerate}
\end{enumerate}
\end{note}


\section{二次函数与一元二次方程、不等式}

\begin{definition}{一元二次不等式}{polynomial-inequality}
只含有一个未知数,且未知数的最高次数是2的不等式,称为\textcolor{third}{一元二次不等式}。一元二次不等式的一般形式为:
\begin{equation}
ax^2+bx+c<0 \mbox{或} ax^2+bx+c>0
\end{equation}
其中$a$,$b$,$c$均为常数,且$a \neq 0$
\end{definition}

\begin{remark}
对于二次函数$y=ax^2+bx+c \left(a \neq 0 \right)$,我们把使$y=ax^2+bx+c=0$的实数$x$叫做二次函数的\textcolor{second}{零点},而实数$x$也正是二次函数与平面直角坐标系$x$轴交点的$x$坐标。
\end{remark}

\begin{table}[htbp]
  \caption{二次函数与一元二次方程、不等式的解的对应关系\label{tab:color thm}}
  \centering
  \begin{tabular}{cccc}
  \toprule
              
              & $\triangle > 0$
              & $\vartriangle = 0$
              & $\vartriangle < 0$\\
  \midrule
              $y=ax^2+bx+c \left(a > 0 \right)$
              & 两个零点($x_1 < x_2$)
              & 一个零点($x_1 = x_2$)
              & 无零点\\
              $ax^2+bx+c= 0 \left(a > 0 \right)$
              & $x=x_1 \mbox{或} x=x_2$
              & $\displaystyle x=x_1 = x_2 = -\frac{b}{2a}$)
              & 无实数根\\
              $ax^2+bx+c>0 \left(a > 0 \right)$
              & $\{ x \mid x<x_1 \mbox{或} x>x_2\}$
              & $\displaystyle \{ x \mid x \neq -\frac{b}{2a}\}$
              & $\mathbb{R} $\\
              $ax^2+bx+c<0 \left(a > 0 \right)$
              & $\{ x \mid x_1 < x < x_2\}$
              & $\varnothing $
              & $\varnothing $\\
  \bottomrule
  \end{tabular}
\end{table}

\begin{problemset}
\item 已知$0<x<1, 0<y<1$, 若$M=xy, N=x+y-1$,则$M$与$N$的大小关系是(\hspace{1cm})
\choice{$M<N$}{$M>N$}{$M=N$}{不确定}
\item 
\end{problemset}