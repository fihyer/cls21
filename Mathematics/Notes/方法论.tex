\chapter{方法论}
\label{ch:方法论}

\section{解题思路的理解和来源}
平时大家评价一个孩子“聪明”或者“不聪明”的依据是看这个孩子对某件事或很多事的反应以及有没有他子集的看法。比如一个“聪明”的孩子,往往反应快、思路清楚,有自己的逐渐。那么我们任务“反应快、思路清晰、有主见”是聪明的前提。学习成绩好的同学,反应快、思路清晰、有主见就是他们的必备条件。\\
那么解题也是如此,必须反应快、思路清晰、有主见。同一道题,不同的同学从不同的角度去理解,由不同的看法最终汇聚成正确的解题过程,这是解题的必然。无论是推导、还是硬性套用、凭借经验做题,都是思路的一种。有的同学由开始思路不清渐渐转变为清楚,有的同学根本没有思路,这就形成了做题上的差距。\\
\section{如何训练解题能力}
训练数学解题思想可以从必要性思维开始。必要性思维也是解答数学试题的万能法门,相对也是最直接、最快捷的答题思想。所谓的必要性思维就是通过所求结论或者某一限定条件寻求前提条件的思维。
\section{从问题入手找条件}
我们发现稍有难度的考题,出题者都设置了种种障碍。如果从已知条件出发,联系所学知识顺退下去也可能解决问题,但是往往应为已知条件和所学知识发生的可能联系太多,从一个个岔路试下去经常越做越复杂,难以得到答案。如果从问题入手,思考想要获得所求所证,必须要做什么,这样

\section{解题表}

% \begin{table}[htbp]
%   \caption{解题表\label{tab:color thm}}
%   \centering
%   \begin{tabular}{cc}
%   \toprule
              
%               & \\
%   \midrule
%               你必须弄清楚问题
%               & \multirow{3}[1]{=}{未知是什么?已知是什么?条件是什么?满足条件是否可能?要确定未知,条件是否充分?或者它是否不充分?或者是多余的?或者是矛盾的?} \\*
              
%                & \multirow{2}[1]{=}画图,引入适当的符号\\*
              
%                & \multirow{2}[1]{=} 把条件的各个部分分开。你能否把他们写下来?\\*
%   \bottomrule
%   \end{tabular}
% \end{table}

\subsection{弄清问题}
\textbf{你必须弄清楚问题}:

