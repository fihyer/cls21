\documentclass[lang=cn,11pt,chinese]{elegantbook}
\usepackage{centernot}
\usepackage{commath}
\usepackage{txfonts}
\usepackage{amsmath}
\usepackage{amssymb}
\usepackage{wrapfig}
\usepackage{bookmark}


\title{数学笔记}
\subtitle{必修第一册}

\author{孙闰泽}
\institute{山东省桓台第二中学}
\date{October 30, 2020}
\version{0.01Beta}
% \bioinfo{自定义}{信息}

\extrainfo{数学是一种精神,一种理性的精神。正是这种精神,激发、促进、鼓舞并驱使人类的思维得以运用到最完善的程度,亦正是这种精神,试图决定性地影响人类的物质、道德和社会生活;试图回答有关人类自身存在提出的问题;努力去理解和控制自然;尽力去探求和确立已经获得知识的最深刻的和最完美的内涵。—— 克莱因}

\logo{pi.jpg}
\cover{mathcover.jpg}

% 本文档命令
\usepackage{array}
\newcommand{\ccr}[1]{\makecell{{\color{#1}\rule{1cm}{1cm}}}}

\begin{document}

\maketitle
\frontmatter

\part{必修一}

\chapter*{前言}
\markboth{Introduction}{前言}
高中各科学习具有知识量大、抽象性高、理论性强、系统性强、综合性强以及能力要求高等方面的特点。凌乱的知识点不利于记忆和掌握。只有把他们穿起来,形成一个体系,才有助于快速高效系统的掌握知识
\vskip 1.5cm

\begin{flushright}
Runze Sun\\
October 30, 2020
\end{flushright}

\tableofcontents
%\listofchanges

\mainmatter

%高一数学必修一
\chapter{集合与常用逻辑用语}
\label{ch:集合与常用逻辑用语}

\begin{introduction}
  \item 集合概念~\ref{def:set}
  \item 集合的性质~\ref{property:set-property}
  \item 集合的表示~\ref{def:rep}
  \item 子集定义~\ref{def:subset}
  \item 真子集定义~\ref{def:proper-subset}
  \item 子集、真子集性质~\ref{property:set-property}
  \item 并集定义~\ref{def:union}
  \item 交集定义~\ref{def:intersection}
  \item 补集定义~\ref{def:complementary}
  \item 集合的运算~\ref{property:set-operation}
  \item 容斥原理~\ref{thm:IO}
\end{introduction}

只要研究问题,就有研究对象。这些研究对象都是数学中的元素,而且数学研究的很多对象又都是元素间具有某些关系的集合。为了简洁、准确的表达数学研究对象以及研究范围,我们需要集合(论)的语言和工具。另一方面,数学具有严谨性,而且它的严谨性是有一系列表示关系的逻辑术语把表示概念的名词链接在一起而体现的,由此,从条件到结论便不会产生歧义,可以被其他人理解。因此为了保证数学的最基本严谨性和准确性,逻辑用语便成了数学表达和交流的另一个基本语言和工具。


\section{集合的概念}

在研究问题、表达交流以及日常交流时,大家在同一个范围、讨论的是同类问题(事情),这样才会又实际效果,否则就会出现风马牛不相及的局面。同样的,研究数学问题是,也学要明确研究对象、确定研究范围、准确表达数学对象及研究范围。

\begin{definition}{集合描述性定义}{set}
  集合是具有某种特性的事物的整体,或是一些确定对象的汇集。构成集合的事物或者对象称为集合的\textcolor{main}{\textbf{元素}}或着成员。 
  \begin{enumerate}
	  \item 集合的元素用小写拉丁字母 $a, b, c, ...$ 表示;
	  \item 集合本身用大写拉丁字母 $A, B, C, ...$ 表示。
  \end{enumerate}
  如果 $a$ 是 $A$ 的元素,就记做\underline{$a \in A$}, 读作 $a$ \textbf{属于} $A$; 如果 $a$ 不是 $A$ 的元素,就记做\underline{$a \notin A$}, 读作 $a$ \textbf{不属于} $A$。
\end{definition}

\begin{remark}
  集合是一个原始的、不定义的概念,它同平面几何中的“点”、“线”、“面”这些概念一样都是指描述性的说明。在理解集合的时候要集合实例体会集合的\textbf{整体性} 和\textbf{广泛性}:
  \begin{enumerate}
    \item \textbf{整体性} : 集合是一个整体,暗指“所有”、“全部”、“全体”的含义,因此一些对象一旦组成了集合,那么这个集合就是这些对象的总体。
    \item \textbf{广泛性} : 集合中的元素可以是任何事物,现实生活中的我们看到的、听到的、问到的、触摸到的、想到的各种各样的事物或一些抽象的符号、数字等等,都可以看作是集合的元素。
  \end{enumerate}
\end{remark}

\begin{property}\label{property:set-property}
集合的特性(集合中元素的特性)
  \begin{enumerate}
    \item \textcolor{main}{\textbf{确定性}} : 给定一个集合,任给一个元素,那么该元素要么属于这个集合要么不属于这个集合,不允许又模棱两可的情况出现。即给定集合 $A$ , 则 $\exists a, a \in A$ 或 $a \notin A$。 
    \item \textcolor{main}{\textbf{互异性}} : 一个集合中任意两个元素都是互不相同的,即集合中的每个元素只能出现一次。即给定集合$A$ 和任意的元素 $a, b \in A$, 则 $a \neq b$。
    \item \textcolor{main}{\textbf{无序性}} : 一个集合中每个元素的地位都是相同的,元素之间没有顺序区别。
  \end{enumerate}
\end{property}

\marginpar{
\begin{note}
  \begin{scriptsize}
  \textcolor{main}{\textbf{有理数}}:正整数、负整数、正分数(正的有限小数、无限小数)、负分数(负的有限小数、无限小数)以及零的统称。\textcolor{main}{\textbf{实数}}: 有理数和无理数的统称。
\end{scriptsize}
\end{note}
}

\begin{table}[htbp]
  \caption{常用数集及其记法\label{tab:color thm}}
  \centering
  \begin{tabular}{cccc}
  \toprule
              数集
              & 意义
              & 记法
              & 示例\\
  \midrule
              非负整数集(或自然数集) 
              & 全体非负整数组成的集合 
              & $\mathbb{N}$
              & $0 \in \mathbb{N}$ \\
              正整数集
              & 全体正整数组成的集合 
              & $\mathbb{N}^*$ 或$\mathbb{N}_+$
              & $0 \notin \mathbb{N}^*$ \\
              整数集
              & 全体整数组成的集合 
              & $\mathbb{Z}$
              & $1 \in \mathbb{Z}$; $0.45 \notin \mathbb{Z}$\\
              有理数集
              & 全体有理数组成的集合 
              & $\mathbb{Q}$
              & $\displaystyle \frac{3}{4} \in \mathbb{Q}$; $\pi \notin \mathbb{Z}$\\
              实数集
              & 全体实数组成的集合 
              & $\mathbb{R}$
              & $\sqrt{2}+1 \in \mathbb{R}$ \\
  \bottomrule
  \end{tabular}
\end{table}


我们可以通过使用语言或者图的方式来表示集合,如果要求不严格我们可以直接使用图或者自然语言来表示(描述)一个集合。但在数学研究中我们通常使用严谨的表示方式: 列举法和描述法。

\begin{definition}{集合的表示法}{rep}
	\begin{enumerate}
	  \item 列举法 : 把集合中的所有元素\underline{一一列举}出来,用逗号“,”分隔开,并且用花括号“$\{\}$”括起来表示的方法叫做列举法。
	  \item 描述法 : 设$A$是一个集合,且给定范围 $D$, $x$ 为集合 $A$ 中的元素, $P(x)$ 为集合 $A$ 中所有元素 $x$ 所具有的\underline{共同特征},那么用\underline{$\{x\in D \mid P(x)\}$}或\underline{$A=\{x \mid P(x), x \in D \}$}或\underline{$A=\{x \mid P(x)$ 且 $ x \in D \}$}的方式表示集合 $A$ 的方法叫做描述法。
	\end{enumerate}
\end{definition}

\begin{note}{描述法注意事项}
	\begin{enumerate}
	  \item 描述法可以概括为:集合 = $\{$ 元素 $\mid$ 共同特征 $\}$
	  \item 在使用描述法表示集合的时候一定要: 
	  \begin{enumerate}
	    \item 将集合中元素的代表符号写出来(用什么符号表示元素);
	    \item 元素所具有的共同特征一定要明确,不能出现未被清晰说明的字母;
	  \end{enumerate}
	\end{enumerate}
\end{note}

\begin{note}{与集合中的元素有关的问题的求解策略}
	\begin{enumerate}
		\item 确定集合中的元素是什么,即集合是\textcolor{main}{数集}还是\textcolor{main}{点集}
		\item 看结合中的元素满足什么\textcolor{main}{共同特征(限制条件)}
		\item 根据限制条件列式求参数的值或确定集合中元素的个数
		\item 检验集合中元素的\textcolor{main}{互异性}
	\end{enumerate}
\end{note}


\section{集合间的基本关系}

\begin{definition}{子集}{subset}
设两个集合$A$,$B$,如果集合$A$中的任意一个元素都是集合$B$中的元素,那么就成集合$A$是集合$B$的\textcolor{third}{\textbf{子集}},记作
\begin{equation}
A \subseteq B \left(\mbox{或} B \supseteq A \right)
\end{equation}
读作“$A$包含于$B$”(或$B$包含$A$)
\end{definition}

\begin{definition}{真子集}{proper-subset}
设两个集合$A$,$B$,如果集合$A \subseteq B $,但存在元素$x \in B$且$x \notin B$,那么就称集合$A$是集合$B$的\textcolor{third}{\textbf{真子集}},记作
\begin{equation}
A \subsetneqq B \left(\mbox{或} B \supsetneqq A \right)
\end{equation}
读作“$A$真包含于$B$”(或$B$真包含$A$)
\end{definition}

\begin{definition}{空集}{emptyset}
\underline{不包含任何元素的集合}叫做\textcolor{third}{空集},记作$\varnothing$
\end{definition}

\begin{property}\label{property:set-property}
子集、真子集性质
	\begin{enumerate}
		\item 任何一个集合都是它本身的子集,即$\forall A, A \subseteq A$
		\item 空集是任何集合的子集,即$\forall A, \varnothing \subseteq A$
		\item 若$A$, $B$, $C$是集合,则:
		\begin{enumerate}
			\item $A=A$
			\item $A \subseteq B, \mbox{且} B \subseteq A \iff A=B$
			\item $A \subseteq B, \mbox{且} B \subseteq A \implies A \subseteq C$
			\item $A \subsetneqq B, \mbox{且} B \subsetneqq A \implies A \subseteq C$
		\end{enumerate}
		\item 若$U$为全集,对于任何的集合$A$,$B$, 下列命题等价:
		\begin{enumerate}
			\item $A \subseteq B$
			\item $A \cap B = A$
			\item $A \cup B = B$
			\item $\complement_{U}{B} \subseteq \complement_{U}{A}$
		\end{enumerate}
		\item 若集合$A$为非空集合,则:
		\begin{enumerate}
			\item $A$的子集的个数为$2^n$
			\item $A$的真子集的个数为$2^{n}-1$
			\item $A$的非空子集的个数为$2^{n}-1$
			\item $A$的非空真子集的个数为$2^{n}-2$
		\end{enumerate}
	\end{enumerate}
\end{property}

%\begin{conclusion}
%\begin{enumerate}
%\item 空集是任何集合的子集,是非空集合的真子集, 即$\forall A, B \neq \empty$,则$\varnothing \subseteq A, \varnothing \subsetneqq B$
%\item 任何一个集合是它本身的子集,即$\forall A, A \subseteq A$
%\item 子集和真子集都有传递性,即:$\forall A, B, C$,如果$A \subseteq B, B\subseteq C$,则$A \subseteq C$
%\end{enumerate}
%\end{conclusion}


\begin{note}{判断两个集合间的关系方法}
	\begin{enumerate}
		\item 对于用描述法表示的集合,先把集合化简后,从表达式中寻找两个集合间的关系
		\item 对于用列举法表示的集合,直接从元素中找两个集合的关系
	\end{enumerate}
\end{note}


\begin{note}{根据两个集合间的关系求参数方法}
已知两个集合之间的关系(参考“判断两个集合间的关系”笔记),将条件转化为元素或者区间端点的关系,利用数轴、韦恩图等可视工具将元素或者区间端点关系准确画出,进而转化为参数所满足的条件后解决求参问题。
\end{note}


\section{集合的基本运算}

\begin{definition}{并集}{union}
由所有属于集合$A$\textcolor{third}{或}属于集合$B$的元素组成的集合$C$,我们称$C$为集合$A$与集合$B$的\textcolor{third}{并集},记作$C = A \cup B$,读作$C$等于$A$并$B$,即
\begin{equation}
C = A \cup B = \{x \mid x \in A, \mbox{或} x \in B\}
\end{equation}
\end{definition}

\begin{definition}{交集}{intersection}
由所有属于集合$A$\textcolor{third}{且}属于集合$B$的元素组成的集合$C$,我们称$C$为集合$A$与集合$B$的\textcolor{third}{交集},记作$C = A \cap B$,读作$C$等于$A$交$B$,即
\begin{equation}
C = A \cap B = \{x \mid x \in A,\mbox{且} x \in B\}
\end{equation}
\end{definition}

\begin{definition}{全集}
如果一个集合含有\textcolor{third}{所研究问题的所有元素},那么就成这个集合为\textcolor{third}{全集},记作$U$。
\end{definition}

\begin{definition}{补集}{complementary}
对于一个集合$A$,由全集$U$中\textcolor{third}{不属于}集合$A$的元素组成的集合$C$,我们称$C$为集合$A$相对于全集$U$的\textcolor{third}{补集},记作$C = \complement_{U}{A}$,即
\begin{equation}
C = \complement_{U}{A} = \{x \mid x \in U,\mbox{且} x \notin A\}
\end{equation}
\end{definition}

\begin{property}\label{property:set-operation}
设$U$为全集,给定任意集合$A, B, C$,则由如下集合运算性质
	\begin{enumerate}
		\item 交换律
		\begin{enumerate}
			\item $A \cap B = B \cap A$
			\item $A \cup B = B \cup A$
		\end{enumerate}
		\item 结合律
		\begin{enumerate}
			\item $(A \cap B) \cap C = A \cap (B \cap C)$
			\item $(A \cup B) \cup C = A \cup (B \cup C)$
		\end{enumerate}
		\item 分配律
		\begin{enumerate}
			\item $A \cap (B \cup C) = (A \cap B) \cup (A \cap C)$
			\item $A \cup (B \cap C) = (A \cup B) \cap (A \cup C)$
		\end{enumerate}
		\item 德.摩根律
		\begin{enumerate}
			\item $\complement_{U}{A \cap B} = \complement_{U}{A} \cup \complement_{U}{B}$
			\item $\complement_{U}{A \cup B} = \complement_{U}{A} \cap \complement_{U}{B}$
		\end{enumerate}
		\item 吸收律
		\begin{enumerate}
			\item $A \cap (A \cup B) = A$
			\item $A \cup (A \cap B) = A$
		\end{enumerate}
		\item 求补律
		\begin{enumerate}
			\item $A \cup \complement_{U}{A} = U$
			\item $A \cap \complement_{U}{A} = \varnothing$
		\end{enumerate}
	\end{enumerate}
\end{property}

%Venn Graph
% Definition of circles
%\def\firstcircle{(0,0) circle (1.5cm)}
%\def\secondcircle{(0:2) circle (1.5cm)}
%\def\thirdcircle{(1:2) circle (1.5cm)}
%
%\colorlet{circle edge}{blue!50}
%\colorlet{circle area}{blue!20}
%
%\tikzset{filled/.style={fill=circle area, draw=circle edge, thick},
%    outline/.style={draw=circle edge, thick}}
%
%\setlength{\parskip}{5mm}
%% Set A and B
%\begin{tikzpicture}
%    \begin{scope}
%        \clip \firstcircle;
%        \fill[filled] \secondcircle;
%    \end{scope}
%    \draw[outline] \firstcircle node {$A$};
%    \draw[outline] \secondcircle node {$B$};
%    \node[anchor=south] at (current bounding box.north) {$A \cap B$};
%\end{tikzpicture}
%
%%Set A or B but not (A and B) also known a A xor B
%\begin{tikzpicture}
%    \draw[filled, even odd rule] \firstcircle node {$A$}
%                                 \secondcircle node{$B$};
%    \node[anchor=south] at (current bounding box.north) {$\overline{A \cap B}$};
%\end{tikzpicture}
%
%% Set A or B
%\begin{tikzpicture}
%    \draw[filled] \firstcircle node {$A$}
%                  \secondcircle node {$B$};
%    \node[anchor=south] at (current bounding box.north) {$A \cup B$};
%\end{tikzpicture}
%
%% Set A but not B
%\begin{tikzpicture}
%    \begin{scope}
%        \clip \firstcircle;
%        \draw[filled, even odd rule] \firstcircle node {$A$}
%                                     \secondcircle;
%    \end{scope}
%    \draw[outline] \firstcircle
%                   \secondcircle node {$B$};
%    \node[anchor=south] at (current bounding box.north) {$A - B$};
%\end{tikzpicture}
%
%% Set B but not A
%\begin{tikzpicture}
%    \begin{scope}
%        \clip \secondcircle;
%        \draw[filled, even odd rule] \firstcircle
%                                     \secondcircle node {$B$};
%    \end{scope}
%    \draw[outline] \firstcircle node {$A$}
%                   \secondcircle;
%    \node[anchor=south] at (current bounding box.north) {$B - A$};
%\end{tikzpicture}
%
%\begin{tikzpicture}
%	\begin{scope}
%		\clip \firstcircle;
%		\clip \secondcircle;
%		\clip \thirdcircle;	
%	\end{scope}
%	\draw[outline] \firstcircle node {$A$};
%    \draw[outline] \secondcircle node {$B$};
%	\draw[outline] \thirdcircle node {$C$};
%\end{tikzpicture}


%集合运算解题方法
\begin{note}{集合运算的方法}
	\begin{enumerate}
		\item 如果集合中的元素是离散的,或者用列举法表示的集合(元素间无法无规律),使用韦恩图进行求解
		\item 如果集合中的元素是连续的实数(通常集合为用描述法表示),使用数轴工具将集合表示出来进行求解,此时要注意端点的情况(是空心还是实心)
	\end{enumerate}
\end{note}


\begin{note}{根据集合的运算求参数的方法}
	\begin{enumerate}
		\item 与不等式有关的集合,一般利用数轴解决,但要注意端点值的取舍
		\item 若结合中元素能一一列举,则一般先用观察法得到集合中元素之间的关系,再列方程(组)求解
		\item 在参数求出后一定要验证集合中元素的互异性,以及题中给出其他的条件
	\end{enumerate}
\end{note}

\subsection{元素的个数}

\begin{definition}{有限集/无限集}{}
	\begin{enumerate}
		\item 含有\textcolor{third}{有限个元素}的集合$A$叫做\textcolor{third}{有限集},有限集$A$中元素的个数用$card\left(A\right)$ 或 $\mid A \mid$来表示
		\item 含有\textcolor{third}{无限个元素}的集合$B$叫做\textcolor{third}{无限集}
	\end{enumerate}
\end{definition}

\begin{theorem}{容斥原理(特殊情况1)}{IO}
对任意两个有限集合$A$, $B$,有
\begin{equation}
card\left(A \cup B \right) = card\left(A\right) + card\left(B\right) - card\left(A \cap B \right)
\end{equation}
或
\begin{equation}
\mid A \cup B \mid = \mid A \mid + \mid B \mid - \mid A \cap B \mid
\end{equation}
\end{theorem}

\begin{theorem}{容斥原理(特殊情况2)}{IO2}
对任意三个有限集合$A$,$B$,$C$, 有
\begin{equation}
\mid A \cup B \cup C \mid = \mid A \mid + \mid B \mid + \mid C \mid - \mid A \cap B \mid - \mid B \cap C \mid + \mid A \cap B \cap C \mid
\end{equation}
\end{theorem}


\section{充分条件与必要条件}

\begin{definition}{命题}{statement}
\textcolor{third}{可以判断真假},类似“若$p$,则$q$”这样的\textcolor{third}{陈述句}叫做\textcolor{third}{命题}。我们一般称$p$为命题的条件,$q$为命题的结论。判断为真的语句叫做真命题,判断为假的语句叫做假命题。
\end{definition}

\begin{definition}{充分、必要条件}{statement}
	\begin{enumerate}
		\item “若$p$,则$q$”为真命题,叫做由$p$推出$q$,记作$p \implies q$,我们说$p$是$q$的充分条件,$q$是$p$的必要条件。
		\item “若$p$,则$q$”为假命题,叫做由$p$推不出$q$,记作$p \centernot\implies q$,我们说$p$不是$q$的充分条件(或$p$是q的非充分条件),$q$不是$p$的必要条件(或$q$是$p$的非必要条件)
	\end{enumerate}
\end{definition}

\begin{definition}{逆命题}{statement}
如果原命题为“若$p$,则$q$”,则“若$q$,则$p$”叫做原命题的\textcolor{third}{逆命题}
\end{definition}

\begin{definition}{充要条件}{iff}
“若$p$,则$q$”和“若$q$,$p$”都是真命题,我们说由$p$通过推理可以得出$q$,而且由$q$通过推理也可以得出$p$,记作则$p \iff q$。我们称$p$是$q$的充分必要条件,简称充要条件。
\end{definition}

\begin{note}
假设$p$是条件,$q$是结论,设$A$、$B$分别为$p$、$q$所描述对象的集合,则有下列定义和推论:
	\begin{enumerate}
	\item 由$p$可以退出$q$,则$p$是$q$的充分条件,则$q$是$p$的必要条件,此时 $A \subseteq B$。即$p \implies q$ 等价于 $A \subseteq B$。若在此前提条件下在还有如下情况:
	\begin{enumerate}
		\item 由$q$可以推出$p$,则$p$与$q$互为充分必要条件(简称“充要条件”),此时 $A=B$。即 $p \implies q, \mbox{且} q \implies p$ 等价于 $A=B$。
		\item 由$q$不可以推出$p$,则$p$是$q$的充分不必要条件,此时$A \subsetneqq B$。即 $p \implies q, \mbox{且} q \centernot \implies p$ 等价于 $A \subsetneqq B$。
	\end{enumerate} 
	\item 由$p$不可以推出$q$,且$q$可以推出$p$,则$p$是$q$的必要不充分条件, 此时$B \subsetneqq A$。即 $p \centernot \implies q, \mbox{且} q \implies p$ 等价于 $B \subsetneqq A$。
	\item 由$p$不可以推出$q$,且$q$不可以推出$p$,则$p$是$q$的既不充分也不必要条件,此时$A \cap B = \varnothing$。即 $p \centernot \implies q, \mbox{且} q \centernot \implies p$  等价于 $A \cap B = \varnothing$。
	\end{enumerate}
\end{note}


\section{全称量词与存在量词}
根据命题的定义,我们知道命题可以判断真假。但是如果陈述句中含有变量,由于不知道变量代表什么(数、范围),因此无法判断真假,所以不是命题。如果用短语对陈述句中的变脸进行(取值范围)限定,我们就可以根据变量所代表的具体内容判断真假。

\begin{definition}{全称量词、存在量词}
对含有变量的陈述句中\textcolor{third}{变量}进行范围限定的短语,我们在数学上称为\textcolor{third}{量词}。根据对变量范围限定的不同量词可以分为全称量词和存在量词:
\begin{enumerate}
\item 对全体变量进行范围限定的量词,我们称为\textcolor{third}{全称量词}, 符号为$\forall$,常用的全称量词有“所有的”,“任意一个”,“一切”,“任给”,“全体”等
\item 对个别或这部分变量进行范围限定的量词,我们称为\textcolor{third}{存在量词},符号为$\exists$,常用的存在量词有“存在一个”,“有些”,“对某些”,“至少有一个”等 
\end{enumerate}
\end{definition}

\begin{definition}{全称量词命题与存在量词命题}{}
\begin{enumerate}
\item 全称量词命题:含有全称量词的命题,记作:$\forall x \in M, p(x)$,读作“对于$M$中的任意一个$x$,$p(x)$都成立”
\item 存在量词命题:含有存在量词的命题,记作:$\exists x \in M, p(x)$,读作“在$M$中的至少存在一个元素$x$,使$p(x)$成立”
\end{enumerate}
其中$M$为变量$x$的取值范围;$p(x)$表示含有变量的语句
\end{definition}

\begin{definition}{命题的否定}
命题的否定就是对这个命题的真值(真或假)进行取反。即命题的否定值否定该命题的结论,因此命题的否定与原命题的真假性相反。
\end{definition}

\begin{remark}
	\begin{enumerate}
		\item 若全称量词命题$\forall x \in M, p(x)$为真命题,则它的否定$\exists x \in M, \neg p(x)$为假命题
		\item 若存在量词命题$\exists x \in M, p(x)$为真命题,则它的否定$\forall x \in M, \neg p(x)$为假命题
	\end{enumerate}
\end{remark}



%第一章题集
\newpage
\begin{problemset}
	\item 集合$A=\{0,2,a\}$, $B=\{1, a^2\}$,若$A \cup B$有5个元素,则$a$的值可能为(\hspace{1cm})
	\choice{0}{1}{2}{3}
	\item 设集合$A=\{x \mid -1 \leq x < 2 \}$, $B=\{x \mid x < a \}$, 若$A \cap B = \varnothing$, 则$a$的取值范围是(\hspace{1cm})
	\choice{$-1<a\leq 2$}{$a>2$}{$a \geq -1$}{$a > -1$}
	\item 已知集合$A=\{x \mid x^2-3x+2=0, x \in \mathbb{R} \}, B=\{x \mid 0<x<5, x \in  \mathbb{R} \}$,则 满足条件$A \subsetneqq C \subsetneqq B$的集合$C$的个数是(\hspace{1cm})
	\choice{1}{2}{3}{4}
	\item 设$S$为实数集$\mathbb{R}$的非空子集,若对任意$x, y \in S$, 都有$x+y, x-y, xy \in S$,则称$S$为封闭集。下列命题正确的是(\hspace{1cm})
	\choice{自然数集$\mathbb{N}$是封闭集}{整数集$\mathbb{Z}$是封闭集}{集合$S=\{a+b\sqrt{2}\mid a, b \in \mathbb{Z}\}$为封闭集}{若$S$为封闭集,且$1\in S$,则$S$一定为无限集}
	\item  已知集合$A=\{x \mid ax^2-2x+1=0\}$只有一个元素, $B=\{x \mid x^2 -3ax + 2 <0\}$, $C=\{x \mid 0 \leq x < 2\}$, 则(\hspace{1cm}) 
	\choice{$A \subsetneqq B$}{$B \subsetneqq A$}{$A = B$}{$A \cap B = \varnothing$}
	\item 已知集合$M=\{x \mid -4<x<2\}, N=\{x \mid x^2-x-6<0\}$,则$M \cap N = $(\hspace{1cm})
	\choice{$\{x \mid -4<x<3\}$}{$\{x \mid -4<x<-2\}$}{$\{x \mid  -2<x<2 \}$}{$\{x \mid 2<x<3 \}$}
	\item 已知集合$A=\{x \mid x^2-5x+6>0 \}, B=\{x \mid x-1<0 \}$,则$A \cap B = \left(\hspace{1cm}\right)$
	\choice{$(-\infty, 1)$}{$\(-2, 1)$}{$(-3, -1)$}{$(3, +\infty)$}
	\item  已知$a, b \in \mathbb{R}$, 若$\displaystyle \{a, \frac{b}{a}, 1 \} = \{a^2, a+b, 0 \}$, 则$a^{2020}+b^{2020}$的值是\rule{1cm}{0.1mm}个元素。
	\item  由实数$x$, $-x \abs{x}$, $\abs{x}$, $\sqrt[]{x^2}$, $-\sqrt[3]{-x^3}$所组成的集合最多含有\rule{1cm}{0.1mm}个元素,最少含有\rule{1cm}{0.1mm}个元素。
\end{problemset}

\chapter{一元二次函数、方程和不等式}
\label{ch:一元二次函数方程和不等式}

\begin{introduction}
  \item 不等式的基本性质~\ref{property:inequality}
  \item 不等式的其他性质~\ref{property:inequality-others}
  \item 基本不等式~\ref{def:MA}
  \item 一元二次不等式~\ref{def:polynomial-inequality}
\end{introduction}



\section{等式性质与不等式性质}

\begin{axiom}{基本事实}{}
\begin{enumerate}
\item 如果 $a-b$ 是正数,那么 $a > b$; 即$a>b \iff a-b>0$;
\item 如果 $a-b$ 等于零,那么 $a = b$; 即$a=b \iff a-b=0$;
\item 如果 $a-b$ 是负数,那么 $a < b$; 即$a<b \iff a-b<0$.
\end{enumerate}
\end{axiom}

\begin{note}{比较大小的方法}
\begin{itemize}
        \item 作差比较法的两种情况:
        \begin{itemize}
		\item 将差进行因式分解转化为几个因式相乘;
		\item 将差式通过配方转化为几个非负实数之和,然后判断正负。
        \end{itemize}
        \item 作商比较法通常适用于两代数式同号的情况。
        \item 构造函数:将要比较的两个数作为一个函数的两个函数值,根据函数的单调性得出大小关系。
\end{itemize}
\end{note}

\vspace{1cm}

\begin{property}\label{property:inequality}
不等式的基本性质
\begin{enumerate}
\item 不等式的对称性:如果 $a>b$ 那么 $b<a$ 如果 $b<a$,那么 $a>b$. 即$a>b \iff b<a$.
\item 不等式的传递性:如果 $a>b$, $b>c$, 那么 $a>c$. 即$a>b, b>c \Longrightarrow a>c$.
\item 不等式的可加性:如果 $a>b$, 那么 $a+c>b+c$. 即$a>b \iff a+c>b+c$.
\item 不等式的可乘性:
\begin{enumerate}
\item 如果 $a>b$, $c>0$, 那么 $ac>bc$. 即$a>b, c>0 \Longrightarrow ac>bc$.
\item 如果 $a>b$, $c<0$, 那么 $ac<bc$. 即$a>b, c<0 \Longrightarrow ac<bc$.
\end{enumerate}
\item 不等式的同向可加性:如果 $a>b$, $c>d$, 那么 $a+c>b+d$. 即$a>b, c>d \Longrightarrow a+c>c+d$.
\item 不等式的同向可乘性:如果 $a>b>0$, $c>d>0$, 那么 $a>c$. 即$a>b, b>d \Longrightarrow ac>bd$.
\item 不等式的可乘方性:
\begin{enumerate}
\item 如果 $a>b>0$, 那么 $a^n>b^n$, 即$a>b>0 \Longrightarrow a^n>b^n (n\in N, n \geq 2)$.
\item 如果 $a>b>0 (n \in N, n \geq 2)$, 那么 $\sqrt[n]{b}>\sqrt[n]{b}$. 即$a>b>0 \Longrightarrow \sqrt[n]{a}>\sqrt[n]{b} (n\in N, n \geq 2)$.
\end{enumerate}
\end{enumerate}
\end{property}

\vspace{1cm}
\begin{property}\label{property:inequality-others}
不等式的其他性质
\begin{enumerate}
\item 倒数性质:
  \begin{enumerate}
        \item $\displaystyle a>b , ab>0 \implies \frac{1}{a} < \frac{1}{b}$.
        \item $\displaystyle \frac{1}{a}<\frac{1}{b}, ab>0 \implies a>b$.
        \item $\displaystyle a<0<b \implies \frac{1}{a}<\frac{1}{b}$.
        \item $\displaystyle a>b>0, 0<c<d \implies \frac{a}{c}>\frac{b}{d}$.
        \item $0<a<x<b$ 或 $\displaystyle a<x<b<0 \implies \frac{1}{b}<\frac{1}{x}<\frac{1}{a}$
  \end{enumerate}
\item 分数性质:若 $a>b>0, m>0$ \\
\begin{enumerate}
        \item 则 $\displaystyle \frac{b}{a} < \frac{b+m}{a+m}; \frac{b}{a} > \frac{b-m}{a-m} (b-m>0)$;\\
        \item 则 $\displaystyle \frac{a}{b} > \frac{a+m}{b+m}; \frac{a}{b} < \frac{a-m}{b-m} (b-m>0)$.
\end{enumerate}
\end{enumerate}
\end{property}


\section{基本不等式}

\begin{definition}{基本不等式}{MA}
如果 $a, b \in \mathbb{R}, a>0, b>0$ 那么
\begin{equation}
\displaystyle \frac{a+b}{2} \geq \sqrt{ab} 
\end{equation}
当且仅当 $a=b$ 时, 等号成立。其中, $\displaystyle \frac{a+b}{2}$ 叫做正数 $a, b$ 的算数平均数,$\sqrt{ab}$ 叫做正数 $a, b$ 的几何平均数。
\end{definition}

\begin{note}
\begin{enumerate}
\item 基本不等式中的$a$与$b$都是实数,如果不是实数,例如已知向量的内积$\vec{a}\cdot \vec{b} = 1$, 则$\vec{a} + \vec{b} \geq 2\sqrt[]{\vec{a} \cdot \vec{b}} = 2$,就是错的。
\item 基本不等式中的$a$与$b$这两个符号可以代表数字,也可以代表代数式(单项式、多项式、整式、分式、指数式、对数式、三角式等等),例如:
\begin{eqnarray*}
\displaystyle x + \frac{2}{x} \geq 2\sqrt[]{2}\\
\displaystyle \frac{2}{x} + \frac{x}{2} \geq 2, (x >0)\\
2^x + 2^y \geq 2\sqrt[]{2^(x+y)}\\
\log_{a}{b} + \log_{b}{a} \geq 2 (\log_{a}{b} > 0)\\
\displaystyle \sin{x} + \frac{1}{\sin{x}} \geq 2 (0< \sin{x} \leq 1)\\
\displaystyle \frac{a^2+b^2}{ab} = \frac{a}{b} + \frac{b}{a} \geq 2 (a, b > 0)\\
\end{eqnarray*}
\item 熟悉理解基本不等是的“正向操作”还要注意利用反向操作,即$\displaystyle \sqrt[]{ab} \leq \frac{a+b}{2}$
\end{enumerate}
\end{note}

\begin{note}{重要不等式}
\begin{enumerate}
        \item $a^2+b^2 \geq 2ab (a,b \in \mathbb{R})$, 当且仅当$a=b$ 时取等号;
        \item \vspace{2mm} $\displaystyle ab \leq \left( \frac{a+b}{2}\right)^2 (a, b\in\mathbb{R})$, 当且仅当 $a=b$ 时取等号;
        \item \vspace{2mm} $\displaystyle \sqrt{a^2+b^2} \geq \left( \frac{a+b}{2}\right)^2 (a,b\in\mathbb{R})$, 当且仅当 $a=b$ 时取等号; 
        \item \vspace{2mm} $\displaystyle min(a, b) \leq \frac{2ab}{a+b} \leq \sqrt{ab} \leq \frac{a+b}{2} \leq \sqrt{\frac{a^2+b^2}{2}} \leq max(a, b) $
\end{enumerate}
\end{note}

\begin{remark}
上述“重要不等式笔记”中:
\begin{enumerate}
\item $\displaystyle \frac{2ab}{a+b}=\frac{2}{\frac{1}{a}+\frac{1}{b}}$,被称为“调和平均数”;
\item $\sqrt{ab}$,被称为“几何平均数”
\item $\frac{a+b}{2}$,称为“算数平均数”
\item $\sqrt{\frac{a^2+b^2}{2}}$,称为“平方平均数”
\end{enumerate}

\end{remark}

\begin{theorem}{和定积最大,积定和最小}
设$a$, $b$都是正数:
\begin{enumerate}
\item 若$a+b=S (S\mbox{为常数})$, 那么当$a=b$时,$\displaystyle (ab)_{max}=\frac{S^2}{4}$
\item 若$ab=P (P\mbox{为常数})$, 那么当$a=b$时,$\displaystyle (a+b)_{min}=2\sqrt[]{P}$
\end{enumerate}
\end{theorem}


\begin{note}{基本不等式的性质}
要利用基本不等式求最值首先要满足一下三个条件:
\begin{enumerate}
\item 一正:相加或着想乘的两个数需为正数;
\item 二定:两个数的和或着积为定值(常数值);
\item 三相等:当且仅当相加或者相乘的这两个数相等的时候基本不等式取等号。
\end{enumerate}
\end{note}

\vspace{3mm}
\begin{note}{利用基本不等式求最值的方法}
\begin{enumerate}
\item 直接解法:如果所给条件满足“一正,二定,三相等”,则直接套用基本不等式。
\item 配凑法:根据所给代数式的“外型”配凑除积、和为常数的形式,然后在利用基本不等式求最值。
\item 换元法:
\item 1的代换法:如果所给条件为分数和定值(整数和定值)形式,让求整数和(分数和)的最值,则将所求(和)式子乘以已知式子并乘以定值的倒数,化简后利用基本不等式求最值。
\item 消元法:
\begin{enumerate}
\item 根据所给条件(一般为方程)建立两个变量之间的(代数)函数关系,将其中一个变量代入所求代数式中
\end{enumerate}
\end{enumerate}
\end{note}


\section{二次函数与一元二次方程、不等式}

\begin{definition}{一元二次不等式}{polynomial-inequality}
只含有一个未知数,且未知数的最高次数是2的不等式,称为\textcolor{third}{一元二次不等式}。一元二次不等式的一般形式为:
\begin{equation}
ax^2+bx+c<0 \mbox{或} ax^2+bx+c>0
\end{equation}
其中$a$,$b$,$c$均为常数,且$a \neq 0$
\end{definition}

\begin{remark}
对于二次函数$y=ax^2+bx+c \left(a \neq 0 \right)$,我们把使$y=ax^2+bx+c=0$的实数$x$叫做二次函数的\textcolor{second}{零点},而实数$x$也正是二次函数与平面直角坐标系$x$轴交点的$x$坐标。
\end{remark}

\begin{table}[htbp]
  \caption{二次函数与一元二次方程、不等式的解的对应关系\label{tab:color thm}}
  \centering
  \begin{tabular}{cccc}
  \toprule
              
              & $\triangle > 0$
              & $\vartriangle = 0$
              & $\vartriangle < 0$\\
  \midrule
              $y=ax^2+bx+c \left(a > 0 \right)$
              & 两个零点($x_1 < x_2$)
              & 一个零点($x_1 = x_2$)
              & 无零点\\
              $ax^2+bx+c= 0 \left(a > 0 \right)$
              & $x=x_1 \mbox{或} x=x_2$
              & $\displaystyle x=x_1 = x_2 = -\frac{b}{2a}$)
              & 无实数根\\
              $ax^2+bx+c>0 \left(a > 0 \right)$
              & $\{ x \mid x<x_1 \mbox{或} x>x_2\}$
              & $\displaystyle \{ x \mid x \neq -\frac{b}{2a}\}$
              & $\mathbb{R} $\\
              $ax^2+bx+c<0 \left(a > 0 \right)$
              & $\{ x \mid x_1 < x < x_2\}$
              & $\varnothing $
              & $\varnothing $\\
  \bottomrule
  \end{tabular}
\end{table}

\begin{problemset}
\item 已知$0<x<1, 0<y<1$, 若$M=xy, N=x+y-1$,则$M$与$N$的大小关系是(\hspace{1cm})
\choice{$M<N$}{$M>N$}{$M=N$}{不确定}
\item 
\end{problemset}

\chapter{函数的概念与性质}
\label{ch:函数的概念与性质}

\begin{introduction}
\item 函数的定义~\ref{def:function}
\item 单调性~\ref{def:monotonicity}
\item 单调性推论~\ref{cor:monotonicity-cor}
\item 复合函数~\ref{def:composite-function}
\end{introduction}


\section{函数的概念及表示}

\begin{definition}{函数}{function}
设$A$, $B$是非空的实数集,如果对于集合$A$中的\textcolor{third}{任意一个数$x$}, 按照\textcolor{third}{某种确定的对应关系$f$}, 在集合$B$中\textcolor{third}{都有唯一确定的数$y$}和它对应,那么就成 $f: A \rightarrow B$ 为从集合 $A$ 到集合 $B$ 的一个 \textcolor{third}{函数}, 记作
\begin{equation}
y = f(x), x \in A
\end{equation}
其中, $x$叫做\textcolor{third}{自变量},自变量$x$的取值范围$A$叫做函数的\textcolor{third}{定义域};与自变量$x$的值对应的$y$\textcolor{third}{值}叫做\textcolor{third}{函数值},函数值的取值范围(集合)$\{f(x) \mid x \in A \}$ 叫做函数的\textcolor{third}{值域}。
\end{definition}

\begin{note}
函数的三要素
\begin{enumerate}
\item 定义域
\item 值域
\item 对应关系
\end{enumerate}
函数的值域是集合$B$的子集。
\end{note}

\subsection{复合函数}
基本的初等函数犹如构成物质的最基本的不可分割的微粒一样。中学阶段的初等函数都可以通过对基本初等函数进行四则运算(加减乘除)后获得,然后我们通过对基本初等函数和初等函数进行“复合”运算可以得到更多的函数。

\begin{definition}{基本初等函数}{}
\begin{enumerate}
\item 常数函数: $f(x) = c, (c为常数)$
\item 幂函数: $f(x) = x^a$
\item 指数函数: $f(x) = a^x, (a>0 \mbox{且} a \neq 1)$
\item 对数函数: $f(x) = \log_{a}{x}, (a>0 \mbox{且} a \neq 1)$
\item 三角函数:
\begin{enumerate}
\item 正弦函数 $f(x) = \sin{x}$
\item 余弦函数 $f(x) = \cos{x}$
\end{enumerate}
\end{enumerate}
\end{definition}

\begin{definition}{初等函数}{}
\begin{enumerate}
\item 一次函数: $f(x) = kx + b, (k \neq 0)$实际上就由下列函数的加减乘除运算得来的
\begin{enumerate}
\item 常数函数: $g(x) = k, (k \neq 0)$和$q(x) = b, (b \in \mathbb{R})$;
\item 幂函数:$p(x) = x$ (幂指数为1);
\end{enumerate} 
即$f(x) = g(x) \cdot p(x) + q(x)$;
\item 二次函数: $f(x) = ax^2 + bx + c, (a \neq 0)$ 则是多个幂函数和常数函数组成
\begin{enumerate}
\item 常数函数: $g(x) = a, (a \neq 0)$,$m(x) = b$ 以及$n(x) = c$;
\item 幂函数: $p(x) = x^2$ 和 $q(x)=x$
\end{enumerate}
即$f(x) = g(x) \cdot p(x) + m(x) \cdot q(x) + n(x)$;
\item 正切函数:$\displaystyle f(x) = \tan{x}, (x \neq \frac{\pi}{2} + k \cdot \pi, k \in \mathbb{Z})$则是由
\begin{enumerate}
\item 正弦函数: $p(x) = \sin(x)$;
\item 余弦函数: $q(x) = \cos{x}$
\end{enumerate} 
\vspace{2mm} 相除获得,即$\displaystyle f(x) = \frac{\sin{x}}{\cos{x}}$; 
\item \vspace{2mm} “双刀函数”:$\displaystyle f(x) = x - \frac{1}{x}$由两个幂函数相减而成;
\item \vspace{2mm} “Nike函数”: $\displaystyle f(x) = x + \frac{1}{x}$由两个幂函数相加而得。
\end{enumerate}
\end{definition}

如果$z$是关于$y$的一个函数,而且$y$是关于$x$的一个函数,那么$z$也是关于$x$的一个函数。其实复合函数就是函数套函数。

\begin{definition}{复合函数}{composite-function}
假设有函数$y, u$
\begin{eqnarray}
y=f(u), (\mbox{定义域:}u \in D_f, \mbox{值域:}y \in M_f) \\
u=g(x), (\mbox{定义域:}x \in D_g, \mbox{值域:}u \in M_g)
\end{eqnarray}
若$M_g \cap D_f \neq \varnothing$,则$\forall x \in \left( M_g \cap D_f \right)$,都有唯一确定的$y$与之对应,则变量$x$与$y$之间通过变量$u$构成一种函数关系$y = f(u) = f(g(x))$,我们称函数$f(g(x))$为\textcolor{third}{复合函数},记作:
\begin{equation}
\begin{array}{l}
y = f \circ g(x) = f(g(x)): D_g \to M_f
\end{array}
\end{equation}
其中$x$叫复合函数的\textcolor{third}{自变量},自变量$x$的取值范围$D=\{ x \mid x \in D_g, \mbox{且} g(x) \in D_f \}$叫复合函数$f(g(x))$的定义域, $u$叫复合函数的\textcolor{third}{中间变量}, $y$叫复合函数的\textcolor{third}{因变量}
\end{definition}

\begin{figure}[htbp]
 \centering
 \begin{tikzpicture}[ele/.style={fill=black,circle,minimum width=.8pt,inner sep=1pt},every fit/.style={ellipse,draw,inner sep=-2pt}]
 
  \node[ele,label=left:$a$] (a1) at (0,11) {};    
  \node[ele,label=left:$b$] (a2) at (0,10) {};    
  \node[ele,label=left:$c$] (a3) at (0,9) {};
  \node[ele,label=left:$d$] (a4) at (0,8) {};

  \node["90:$g${:}$A \to B$"] (g1) at (3,6) {};
  \node["90:$f${:}$B \to C$"] (f1) at (7,6) {};
  \node["90:$f \circ g${:}$A \to C$"] (f1) at (5,5) {};
  
  \node[ele,label=left:$a$] (m1) at (0,4) {};    
  \node[ele,label=left:$b$] (m2) at (0,3) {};    
  \node[ele,label=left:$c$] (m3) at (0,2) {};
  \node[ele,label=left:$d$] (m4) at (0,1) {};
    
  \node[ele,,label=90:$1$] (b1) at (5,11) {};
  \node[ele,,label=90:$2$] (b2) at (5,10) {};
  \node[ele,,label=90:$3$] (b3) at (5,9) {};
  \node[ele,,label=90:$4$] (b4) at (5,8) {};

  \node[ele,,label=right:$1$] (n1) at (9,4) {};
  \node[ele,,label=right:$9$] (n2) at (9,3) {};
  \node[ele,,label=right:$4$] (n3) at (9,2) {};
  \node[ele,,label=right:$16$] (n4) at (9,1) {};
  
  \node[ele,,label=right:$1$] (c1) at (9,11) {};
  \node[ele,,label=right:$9$] (c2) at (9,10) {};
  \node[ele,,label=right:$4$] (c3) at (9,9) {};
  \node[ele,,label=right:$16$] (c4) at (9,8) {};

  \node[draw,label=90:$A$, fit= (a1) (a2) (a3) (a4),minimum width=2cm] {} ;
  \node[draw,label=90:$B$, fit= (b1) (b2) (b3) (b4),minimum width=2cm] {} ; 
  \node[draw,label=90:$C$, fit= (c1) (c2) (c3) (c4),minimum width=2cm] {} ; 
   
  \node[draw,label=90:$A$, fit= (m1) (m2) (m3) (m4),minimum width=2cm] {} ;
  \node[draw,label=90:$C$, fit= (n1) (n2) (n3) (n4),minimum width=2cm] {} ; 

  \draw[->,thick,shorten <=2pt,shorten >=2pt] (a1) -- (b4) -- (c4);
  \draw[->,thick,shorten <=2pt,shorten >=2] (a2) -- (b2) -- (c3);
  \draw[->,thick,shorten <=2pt,shorten >=2] (a3) -- (b1) -- (c1);
  \draw[->,thick,shorten <=2pt,shorten >=2] (a4) -- (b3) -- (c2);
  
  \draw[->,thick,shorten <=2pt,shorten >=2pt] (m1) -- (n4);
  \draw[->,thick,shorten <=2pt,shorten >=2] (m2) -- (n3);
  \draw[->,thick,shorten <=2pt,shorten >=2] (m3) -- (n1);
  \draw[->,thick,shorten <=2pt,shorten >=2] (m4) -- (n2);
 \end{tikzpicture}
\end{figure}

\begin{note}
    \begin{enumerate}
        \item 复合函数$f(g(x))$中我们也可以称$y=f(u)$为外函数,$u=g(x)$为内函数,且内函数的值域必须是外函数的定义域的子集。
        \item 给出$y=\mbox{解析式}$形式的复合函数时可以根据\textcolor{third}{基本初等函数}(5个)和\textcolor{third}{初等函数}(3个)“外型”,将各类函数中自变量$x$所在位置看作一个占位符号(例如$\sqcup$)进行合理拆分:
        \begin{enumerate}
            \item 基础初等函数:
            \begin{enumerate}
                \item 幂函数: $f(\sqcup) = {\sqcup}^a$
                \item 指数函数: $f(\sqcup) = a^{\sqcup}, (a>0 \mbox{且} a \neq 1)$
                \item 对数函数: $f(\sqcup) = \log_{a}{\sqcup}, (a>0 \mbox{且} a \neq 1)$
                \item 三角函数:
                \begin{enumerate}
                    \item 正弦函数 $f(\sqcup) = \sin{\sqcup}$
                    \item 余弦函数 $f(\sqcup) = \cos{\sqcup}$
                \end{enumerate}
            \end{enumerate}
            \item 初等函数:
            \begin{enumerate}
                \item 一次函数: $f(\sqcup) = k{\sqcup} + b, (k \neq 0)$
                \item \vspace{2mm} 二次函数: $f(\sqcup) = a{\sqcup}^2 + b{\sqcup} + c, (a \neq 0)$ 
                \item \vspace{2mm} 正切函数:$\displaystyle f(\sqcup) = \tan{\sqcup}, ({\sqcup} \neq \frac{\pi}{2} + k \cdot \pi, k \in \mathbb{Z})$
            \end{enumerate}
        \end{enumerate}
    \end{enumerate}
\end{note}

\vspace{3mm}
\begin{note}
\begin{enumerate}
\item 对于复合函数$f(g(x))$,其定义域仍然是求自变量$x$的取值范围,而不是$g(x)$的范围。
\item 对于相同对应关系下的函数$f(x), f(g(x)), \mbox{以及} f(h(x))$, 其中对应的$x, g(x) \mbox{以及} h(x)$的范围相同。
\end{enumerate}
\end{note}

\vspace{3mm}
\begin{note}
求定义域
\begin{enumerate}
\item 已知函数$f(x)$的定义域为$I$,求$f(g(x))$的定义域:实质为$g(x) \in I$, 根据$g(x)$的解析是求的出$x$的范围,即为函数$f(g(x))$的定义域;
\item 已知函数$f(g(x))$的定义域为$I$,求$f(x)$的定义域:实质为$x \in I$, 根据$g(x)$的解析是求的出$g(x)$的范围,即为函数$f(x)$的定义域;
\item 已知函数$f(g(x))$的定义域为$I$,求$f(h(x))$的定义域:实质为$x \in I$, 先根据$g(x)$的解析是求出$g(x)$的范围(即为函数$f(x)$的定义域),然后将其作为$h(x)$的范围,再根据$h(x)$的解析式求出$x$的范围。
\end{enumerate}
\end{note}

\subsection{函数图像的变换}

通过一个函数图像进行适当的变换得到另一个与之有关的函数的图像,我们称之为图像的变换。图像变换的实质其实就是图像\textcolor{third}{坐标的替换}

\begin{conclusion}{图像的对称变换}{}
\begin{enumerate}
\item \vspace{2mm} 函数$f(\alpha)$与函数$f(\beta)$关于直线$\displaystyle x=\frac{\alpha + \beta}{2}$对称:
\begin{enumerate}
\item \vspace{2mm} (\textbf{两个函数对称})函数 $f(x)$ 与 函数$f(2a-x)$ 的图象关于直线 $x=a$ 对称;其中 $\displaystyle \frac{x+ (2a -x)}{2}= a$;
\item \vspace{2mm} (\textbf{两个函数对称})函数 $f(a+x)$ 与 函数$f(a-x)$ 的图象关于直线 $x=a$对称($\displaystyle \frac{(a+x)+(a-x)}{2}=a$);
\item \vspace{2mm} (\textbf{两个函数对称})函数 $f(x-a)$ 与 函数$f(a-x)$ 的图象关于 $y$ 轴对称(关于$y$轴对称也即是关于直线$x=0$对称,其中$\displaystyle \frac{(x-a)+(a-x)}{2}= 0 )$;
\item \vspace{2mm} 函数 $f(x)$ 与 $f(-x)$ 的图象关于 $y$ 轴对称;
\end{enumerate}
\item \vspace{2mm} 函数 $f(x)$ 与 $-f(x)$ 的图象关于 $x$ 轴对称;
\item \vspace{2mm} 函数 $y = m + f(\alpha)$与 函数$y = n - f(\beta)$ 的图象关于点 $\displaystyle (\frac{\alpha + \beta}{2}, \frac{m+n}{2})$中心对称:
\begin{enumerate}
\item \vspace{2mm} (\textbf{两个函数对称})$0+f(x)$与$2b-f(2a-x)$关于点$\displaystyle (a=\frac{x+(2a-x)}{2}, b=\frac{0+2b}{2})$中心对称;
\item \vspace{2mm} (\textbf{一个函数对称}) 若函数$y=f(x)$的定义域为$I$, $\forall x \in I$ 满足 $f(a+x)=2b-f(a-x) \implies$函数$f(x)$的图像关于点$(a, b)$中心对称;
\item \vspace{2mm} (\textbf{两个函数对称})函数 $f(x)$ 与 $-f(-x)$ 的图象关于原点$\left(( 0, 0 \right)$中心对称;
\item \vspace{2mm} (\textbf{一个函数对称})函数$y=f(x)$的定义域为$I$, $\forall x \in I$, 都有$-x \in I$,且$f(x) = -f(-x)$(奇函数的定义),则函数$y=f(x)$的图象关于原点$\left(( 0, 0 \right)$中心对称;
\end{enumerate}
\item \vspace{2mm} 函数 $f(x)$ 与 $f^{-1}(x)$ ($f^{-1}(x)$是$f(x)$的反函数)的图象关于直线 $y=x$ 对称;
\end{enumerate}
\end{conclusion}

\vspace{3mm}
\begin{conclusion}{图像的平移变换}{}
\begin{enumerate}
\item 将函数 $f(x)$ 的图像沿 $x$ 轴平移 $\mid a \mid, a \in \mathbb{R}$ 个单位,得到函数$f(x+a)$的图像。$a>0$时向左移($x$轴的负半轴方向),$a<0$时向右移($x$轴的正半轴方向)。
\item 将函数 $f(x)$ 的图像沿 $y$ 轴平移 $\mid a \mid, a \in \mathbb{R}$ 个单位,得到函数$f(x) + a$的图像。$a>0$时向上平移($y$轴的正半轴方向),$a<0$时向下平移($y$轴的负半轴方向)。
\end{enumerate}
\textbf{记忆:} 
\begin{enumerate}
\item \textcolor{main}{自变量}坐标替换:“\textbf{图像左加右减}”
\item \textcolor{main}{因变量}坐标替换:“\textbf{图像上加下减}”
\end{enumerate}
\end{conclusion}

\vspace{3mm}
\begin{conclusion}{图像的伸缩变换}
\begin{enumerate}
\item 将 $f(x)$ 的图象上各点的纵坐标伸长( $a>1$ )或缩短( $0<a<1$ )到原来的 $a$ 倍,而横坐标不变,得到函数 $af(x), (a>0)$ 的图象;
\item 将 $f(x)$ 的图象上各点的横坐标伸长( $0<a<1$ )或缩短( $a>1$ )到原来的 $\displaystyle \frac{1}{a}$ 倍,而纵坐标不变,得到函数 $f(ax), (a>0)$ 的图象.
\end{enumerate}
\textbf{记忆:} 
\begin{enumerate}
\item \textcolor{main}{自变量}坐标替换:“\textbf{图像分伸整缩}”
\item \textcolor{main}{因变量}坐标替换:“\textbf{图像整升分降}”
\end{enumerate}
\end{conclusion}

\vspace{3mm}
\begin{conclusion}{图像的翻折变换}
\begin{enumerate}
\item 将函数 $f(x)$ 的图像中在 $x$ 轴下方的部分,沿$x$轴对折到$x$上方得到$\mid f(x) \mid$的图像。
\item 将函数 $f(x)$ 的图像中在 $x$ 轴正半轴的图像沿着$y$轴对折到$y$轴左侧,得到$f(\mid x \mid)$的图像。
\end{enumerate}
\textbf{记忆:} 
\begin{enumerate}
\item \textcolor{main}{自变量}坐标替换:“\textbf{图像右翻左}”,
\item \textcolor{main}{因变量}坐标替换:“\textbf{图像下翻上}”。
\end{enumerate} 
\end{conclusion}

\subsection{抽象函数}
\begin{table}[htbp]
  \caption{常用具体函数对应的抽象函数\label{tab:color thm}}
  \centering
  \begin{tabular}{cc}
  \toprule
              抽象函数的性质
              & 对应的具体函数模型\\
  \midrule
              $f(x) \pm f(y) = f(x \pm y)$ 
              & $f(x)=kx+b, (k \neq 0)$\\
       		  \midrule
              $f(x) \cdot f(y) = f(x \cdot y)$\\
              $\displaystyle \frac{f(x)}{f(y)} = f(\frac{x}{y})$
              & $f(x) = x^a $\\
              \midrule
              $f(x) \cdot f(y) = f(x + y)$\\
              $\displaystyle \frac{f(x)}{f(y)} = f(x - y)$
              & $f(x) = a^x, (a >0 \mbox{且} a \neq 1) $\\
              \midrule
              $f(x + y) = f(x) \cdot f(y)$\\
              $\displaystyle f(x - y) = \frac{f(x)}{f(y)}$
              & $f(x)=\log_{a}{x}, (a>0 \mbox{且} a \neq 1)$\\
              \midrule
              $f(x+y) = f(x)\cdot g(y) + g(x)\cdot f(y) $
              & $f(x) = \sin{x}, g(x)=\cos{x} $\\
              $\displaystyle f(x+y) = \frac{f(x)+f(y)}{1-f(x)f(y)}$
              & $f(x)=tanx$\\
  \bottomrule
  \end{tabular}
\end{table}

\section{函数的基本性质}

\subsection{函数的单调性与最值}

\begin{definition}{单调性定义}{monotonicity}
设函数 $f(x)$ 的定义域为 $I$, 区间 \underline{$D \subseteq I$}, 如果 \underline{$\forall x_1, x_2 \in D$}, 当 \underline{$x_1 < x_2$ }时, 都有\underline{$f(x_1)<f(x_2)$ }(或者 $f(x_1)>f(x_2)$),那么就称 $f(x)$ 在\underline{区间 $D$ }上\underline{\textbf{单调递增}} (或者\textbf{单调递减})\\
区间 $D$ 称为函数 f(x) 的\underline{\textbf{单调递增区间}} (或者 \textbf{单调递减区间}).
\end{definition}

\begin{corollary}{单调性推论}{monotonicity-cor}
设函数 $f(x)$ 的定义域为 $I$, 区间 \underline{$D \subseteq I$}, $\bigtriangleup x = x_1 - x_2, (x_1 \neq x_2), \bigtriangleup y = f(x_1) - f(x_2)$,则:
\begin{enumerate}
\item \vspace{2mm} 函数$f(x)$在区间$D$上\textcolor{third}{单调递增} $\iff \forall x_1, x_2 \in D, x_1 \neq x_2, \mbox{都有} \displaystyle \frac{\bigtriangleup y}{\bigtriangleup x} > 0$
\item \vspace{2mm} 函数$f(x)$在区间$D$上\textcolor{third}{单调递减} $\iff \forall x_1, x_2 \in D, x_1 \neq x_2, \mbox{都有} \displaystyle \frac{\bigtriangleup y}{\bigtriangleup x} < 0$
\end{enumerate}
\end{corollary}


\begin{table}[htbp]
  \caption{函数单调性运算\label{tab:color thm}}
  \centering
  \begin{tabular}{cccc}
  \toprule
              运算
              & 函数$f(x)$单调性
              & 函数$g(x)$单调性
              & 结果\\
  \midrule
              $+$ 
              & $\uparrow $
              & $\uparrow $
              & $\uparrow $\\
              $+$
              & $\downarrow $
              & $\downarrow $
              & $\downarrow $\\
              $-$
              & $\uparrow $
              & $\downarrow $
              & $\uparrow $\\
              $-$
              & $\downarrow $
              & $\uparrow $
              & $\downarrow $\\
              $ \times $
              & $\uparrow $
              & $\uparrow $
              & $\uparrow $\\
              $ \times $
              & $\downarrow $
              & $\downarrow $
              & $\downarrow $\\
  \bottomrule
  \end{tabular}
\end{table}



%\begin{note}{抽象函数的单调性}
%\end{note}


\begin{definition}{函数的最值}
设函数 $f(x)$ 的定义域为 $I$, 如果 $\exists M \in \mathbb{R}$ 满足:
\begin{enumerate}
\item $\forall x \in I $, 都有 $f(x) \leq M$  (或者 $f(x) \geq M$)
\item $\exists x_0 \in I$ , 使得 $f(x_0)=M$
\end{enumerate}
那么,我们称 $M$ 是函数 $y=f(x)$ 的 \textbf{最大值} (或者\textbf{最小值})
\end{definition}

\subsection{函数的奇偶性}

\begin{definition}{奇偶性定义}{parity}
设函数$f(x)$的定义域为$I$,如果$\forall x \in I$,都有 $-x \in I$,且$f(x)=f(-x)$(或$f(x)=-f(-x)$),则我们称函数$f(x)$为\textcolor{third}{偶函数}(或\textcolor{third}{奇函数})
\end{definition}

\subsection{函数的周期性}

\begin{definition}{周期性}{cyclicity}
\end{definition}


\section{幂函数}

\begin{definition}{幂函数}{power-function}
\textcolor{second}{形如}$f(x) = x^{\alpha}$这样的函数,我们称之为\textcolor{third}{幂函数}。其中$x$是自变量,$\alpha$是常数。
\end{definition}

\begin{note}
\begin{eqnarray*}
\mbox{幂函数} = \mbox{系数} \cdot \mbox{底数}^{\mbox{幂指数}}
\end{eqnarray*}
其中$\mbox{\textcolor{third}{系数}}=1$, \textcolor{third}{底数}为自变量,\textcolor{third}{幂指数}为常数。
\end{note}


\begin{figure}[htbp]
\centering
     \begin{tikzpicture}
      \begin{axis}[
      	  width=10cm,
      	  xmin=-10,xmax=10,
          ymin=-10,ymax=10,
          axis lines = center,
          legend pos = south east,
          xlabel = $x$,
          ylabel = {$y$},
          title={\mbox{常见的五个幂函数}}
      ]
      \addplot [
          domain=-7:7, 
          samples=100, 
          color=black,
      ]
      {x^(-1)};
      \addplot [
          domain=0:7, 
          samples=100, 
          color=green,
      ]
      {sqrt(x)};
      \addplot [
          domain=-7:7, 
          samples=100, 
          color=orange,
      ]
      {x};
      \addplot [
          domain=-5:5, 
          samples=100, 
          color=red,
      ]
      {x^2};
      \addplot [
          domain=-5:5, 
          samples=100, 
          color=blue,
      ]
      {x^3};
      \node[label={0:{(1,1)}},circle,fill,inner sep=1pt] at (axis cs:1,1) {};
      \node[pin=-45:{$f(x)=x^{-1}$}] at (axis cs:-0.6,-3){};
      \node[fill=white, text=green] at (axis cs:8,2.6){$f(x)=x^{\frac{1}{2}}$};
      \node[fill=white, text=orange] at (axis cs:-4.8,-5){$f(x)=x$};
      \node[fill=white, text=red] at (axis cs:-2.8,8){$f(x)=x^{2}$};
      \node[fill=white, text=blue] at (axis cs:-2.5,-8){$f(x)=x^{3}$};

      %\addlegendentry{$f(x)=x^{-1}$}     
      %\addlegendentry{$f(x)=x^{\frac{1}{2}}$}
      %\addlegendentry{$f(x)=x$} 
      %\addlegendentry{$f(x)=x^{2}$}
      %\addlegendentry{$f(x)=x^{3}$}
      \end{axis}
    \end{tikzpicture}
\end{figure}



\section{函数的应用(一)}

\chapter{指数函数与对数函数}
\label{ch:指数函数与对数函数}

\begin{introduction}
  \item 零点存在定理~\ref{thm:Bolzano-Cauchy1}
\end{introduction}

指数函数与对数函数是两类重要的、应用广泛的基本初等函数。

\section{指数}

\section{指数函数}

\subsection{指数函数的概念}

\subsection{指数函数的图像和性质}

\marginpar{
\begin{note}
  \begin{scriptsize}
  现代指数的概念是于1637由Rene Descartes引入的,他给乘方数设计了专门的记号系统,即指数函数。而约翰·纳皮尔(John Napier)早在20多年前的1614年就发明了对数,他还以一己之力编写了历史上第一张对数表。当时的欧洲,随着天文学和航海的发展,人们处理的数字越来越大,计算越来越复杂,社会上有很强的简化计算的需求,作为计算工具而被发明的对数,不过是回应了时代的呼唤应运而生而已。
\end{scriptsize}
\end{note}
}

\section{对数}

在数学的发展历史上,先有对数,然后才有了指数幂函。如果从运算的角度看,加、减是一级运算,乘、除是二级运算,乘方、开方是三级运算。运算数量级的不同据定了运算的复杂度,一般来说运算的数量级越高,运算的复杂的也就越高。对数可以化乘除为加减,化乘方开方为乘除,将高级运算降为次级运算。因此在对数发明出来的那个年代有一句话叫做:对数的的发明让天文学家增加了一倍的寿命!对数不仅仅在输的运算中占有重要地位,而且也是学习对数函数的基础。

\begin{definition}{对数的概念}{log}
如果 $a^x=N$ $(a \geq 0$, 且 $a \neq 1)$, 那么数 $x$ 叫做以 \underline{$a$ 为\textbf{底}} $N$的对数。记作:
\begin{equation}
x=\log_{a}{N}
\end{equation}
其中 $a$ 叫做对数的底数, $N$ 叫做真数。
\end{definition}

\begin{note}
对比回忆指数:
\begin{enumerate}
\item 指数是计算从1开始,按照某个固定的增长速率$\alpha$,“增长”一段时间$t$之后的总量$M$。用数学符号记作:$M=\alpha^t$
\item 那么对数就是计算以某个固定的增长速率$\alpha$,从1开始“增长”到一个固定量$M$所需要的时间$t$。用数学符号记作: $t=\log_{\alpha}{M}$
\end{enumerate}
例如,从1开始以两倍的速度即$2/s$增长,当到3秒的时候得到的值就是8,也即是$8=2^3$。如果以两倍的速度增长,时间是零,那么得到的值就是1(即$1=2^0$,没有任何增长),如果增长了一段时间之后变成了分数例如变成了$ \frac{1}{2}$,那么也就是说在达到1之前$-1$秒的时候的量是$\frac{1}{2}$(因为从$ \frac{1}{2}$开始以两倍的速度增长一秒就是1), 即从现在的$0$为起点的时间角度看,从起始量1增长了$-1$秒(倒退1秒)得到的量就是$\frac{1}{2} = 2^{-1}$
\end{note}
\section{对数函数}

\subsection{对数函数的概念}

\subsection{对数函数的图像和性质}

\subsection{不同函数增长的差异}

一次函数$f(x)=kx+b, (k>0)$、幂函数$f(x)=x^n, (n>0)$、指数函数$f(x)=a^x, (a>1)$ 以及对数函数$f(x)=\log_{a}{x}, (a > 1)$ 在区间 $(0, +\infty)$上都是增函数。


     \begin{tikzpicture}
      \begin{axis}[
      	  xmin=0,xmax=5,
          ymin=-2,ymax=7,
          axis lines = center,
          legend pos = south west,
          xlabel = $x$,
          ylabel = {$y$},
      ]
      %Below the red parabola is defined
      \addplot [
          domain=-1:5, 
          samples=100, 
          color=blue,
      ]
      {2*x};
      \addplot [
          domain=-2:5, 
          samples=100, 
          color=red,
      ]
      {2^x};

      \node[label={180:{(1,2)}},circle,fill,inner sep=1pt] at (axis cs:1,2) {};
      \node[label={180:{(2,4)}},circle,fill,inner sep=1pt] at (axis cs:2,4) {};
      \addlegendentry{$2x$}      
      \addlegendentry{$2^x$}
      \end{axis}
    \end{tikzpicture}


    \begin{tikzpicture}
      \begin{axis}[
      	  xmin=0,xmax=50,
          ymin=-4,ymax=5,
          axis lines = center,
          xlabel = $x$,
          ylabel = {$y$},
      ]
      %Below the red parabola is defined

      \addplot [
          domain=-40:40, 
          samples=100, 
          color=blue,
      ]
      {(1/10)*x};
      \addplot [
          domain=0:60, 
          samples=100, 
          color=green,
      ]
      {log10(x)};
      
      \addlegendentry{$\frac{1}{10}_x$}      
      \addlegendentry{$\ln{x}$}
      \end{axis}
    \end{tikzpicture}
\section{函数的应用(二)}

回忆一元二次函数、方程与不等式章节,我们用二次函数的观点分别讨论了一元二次方程以及一元二次不等式,也总结了一元二次不等式的\underline{解集端点}就是其对应的一元二次方程的\underline{实数根},同时是其对应的一元二次函数的\underline{零点}。\\

对于$x^2+4x-8=0$这样的方程,我们可以通过公式法求解,但是如果是类似$log_{2}{x}+2x-6=0$,或者是$e^{x-1}+4x-4=0$这样的方程,
\subsection{函数的零点与方程的解}

\begin{theorem}{函数零点存在}{Bolzano-Cauchy1}
如果函数$y=f(x)$在区间$\left[a,b\right]$上的图像是\underline{\textbf{一条连续不断的曲线}},且有$f(a)f(b)<0$,那么,函数$y=f(x)$在区间$\left(a,b\right)$内上\underline{\textbf{至少有一个零点}},即$\exists c \in \left(a,b\right)$,使得$f(c)=0$,这个$c$就是方程$f(x)=0$的解。
\end{theorem}

\marginpar{
\begin{note}
  \begin{scriptsize}
零点存在定理(Bolzano-Cauchy 第一定理)在数学分析上是“闭区间上连续函数的介值定理(Bolzano-Cauchy 第二定理)”的特例,是捷克数学佳波尔查诺(Bolzano)在1817年首先证明的。但是由于但是缺乏实数理论,证明不严格,后来又由德国数学家魏尔施特拉斯(Weierstrass)将这个证明严密化。
  \end{scriptsize}
\end{note}
}

\subsection{二分法}


\chapter{三角函数}
\label{ch:三角函数}

\begin{introduction}
  \item 任意角的概念~\ref{def:any-angle}
  \item 象限角及轴线角概念~\ref{def:quadrant-angle}
  \item 终边相同的角~\ref{def:angle-set}
  \item 终边相同角推论~\ref{cor:angle-set-col}
  \item 圆心角的弧度~\ref{def:central-angle-radian}
  \item 角度弧度转换
\end{introduction}

\section{任意角和弧度制}

\subsection{任意角}

\begin{remark}{\vspace{0.3mm}角的概念及表示}
\begin{enumerate}
\item 角可以看成平面内\underline{一条射线}绕着它的\underline{端点}旋转所成的图形。
\item 射线的端点叫做角的\underline{顶点}
\item 射线的起始位置叫做角的\underline{始边}
\item 射线的终止位置叫做角的\underline{终边}
\end{enumerate}
%\begin{firute}[h]
\begin{wrapfigure}{r}{0.25\textwidth}
\begin{center}
\begin{tikzpicture}
  \draw
    (1.5,0) coordinate (A) node[right] {A}
    -- (0,0) coordinate (O) node[left] {O}
    -- (-1,1) coordinate (B) node[above left] {B}
    pic["$\alpha$", draw=orange, ->, angle eccentricity=1.2, angle radius=0.4cm]
    {angle=A--O--B};
\end{tikzpicture}	
\end{center}
\end{wrapfigure}
%\end{figure}
如图所示$O$为$\angle{AOB}$的顶点,$OA$为$\angle{AOB}$的始边,$OB$为$\angle{AOB}$的终边。
\end{remark}


\begin{definition}{任意角}{any-angle}
\begin{enumerate}
\item \textbf{正角}:按\textcolor{third}{逆时针}方向旋转形成的角
\item \textbf{负角}:按\textcolor{third}{顺时针}方向旋转形成的角
\item \textbf{零角}:\textcolor{third}{没有作任何旋转}的角
\end{enumerate}
\end{definition}


\begin{note}
设$\alpha$, $\beta$是任意两个角,如果把角$\alpha$的终边旋转角$\beta$,则最终形成的角为角$\alpha + \beta$。
\end{note}


\begin{definition}{象限角和轴线角}{quadrant-angle}
在平面直角坐标系中,将角的顶点与原点重合,角的始边与$x$轴的非负半轴重合:
\begin{enumerate}
\item 终边在坐标轴上的角,不属于任何一个象限,我们称之为\textcolor{third}{轴线角};
\item 终边不在坐标轴上的角我们称之为\textcolor{third}{象限角}。角的终边在第几象限,我们就叫这个角是第几象限角。
\end{enumerate}
\end{definition}


\begin{remark}
\begin{enumerate}
\item 锐角: 大于$0^{\circ}$小于$90^{\circ}$的角。锐角一定是第一象限角,但第一象限角不一定是锐角。
\item 直角: 等于$90^{\circ}$的角,直角是轴线角,其终边在$y$轴的非负半轴。
\item 钝角: 大于$90^{\circ}$小于$180^{\circ}$的角。钝角一定是第二象限角,但第二象限角不一定是钝角,即若存在命题$p$, $q$分别为:
\begin{enumerate}
\item $p$:角 $\alpha $ 是钝角,
\item $q$:角 $\alpha $ 是第二象限角,
\end{enumerate}
则$p \implies q, q \centernot \implies p$,或我们说$p$是$q$的充分不必要条件。
\item 平角: 等于$180^{\circ}$的角,平角是轴线角,其终边在$x$轴的非正半轴。
\item 周角: 等于$360^{\circ}$的角,周角是终边在$x$轴的非负半轴的轴线角。
\end{enumerate}
\end{remark}


\begin{definition}{终边相同的角}{angle-set}
任意角$\alpha$终边相同的角$\beta \in S$
\begin{equation}
S = \{\beta \mid \beta = \alpha + k \cdot 360^{\circ}, k \in \mathbb{Z} \}
\end{equation}
\end{definition}

\begin{corollary}{终边相同的角}{angle-set-col}
\begin{enumerate}
\item 终边相同的角之间相差$360^{\circ}$的整数倍。
\item 终边在同一直线上的角之间相差$180^{\circ}$的整数倍。
\item 终边相互垂直的两条直线上的角之间相差$90^{\circ}$的整数倍。
\end{enumerate}
\end{corollary}

\begin{note}{判断象限角的方法}
\begin{enumerate}
\item 图像法: 根据已知条件在平面直角坐标系中画出相应的角,根据象限角的定义观察终边的位置,确定其为第几象限角。
\item 解析法: 先将角$\alpha$转化为$\alpha = k \cdot 360^{\circ} + \beta \left(k \in \mathbb{Z}, \mbox{且} 0^{\circ} < \beta < 360^{\circ} \right)$的形式;然后判断角$\beta$的终边所在的象限; 角$\beta$是第几象限角,角$\alpha$就是第几象限角。
\end{enumerate}
\end{note}

\begin{note}
\begin{table}[htbp]
  \caption{象限角与轴线角的集合\label{tab:color thm}}
  \centering
  \begin{tabular}{ll}
  \toprule
              分类
              & 集合\\
  \midrule
              终边在$x$轴
              & $\{\alpha \mid \alpha = k \cdot 180^{\circ}, k \in \mathbb{Z}\}$\\
              终边在$x$轴非负半轴
              & $\{\alpha \mid \alpha = k \cdot 360^{\circ}, k \in \mathbb{Z}\}$\\
              终边在$x$轴非正半轴
              & $\{\alpha \mid \alpha = k \cdot 360^{\circ} + 180^{\circ}, k \in \mathbb{Z}\}$ \\
              终边在$y$轴
              & $\{\alpha \mid \alpha = k \cdot 180^{\circ} + 90^{\circ}, k \in \mathbb{Z}\}$\\
              终边在$y$非负半轴
              & $\{\alpha \mid \alpha = k \cdot 360^{\circ} + 90^{\circ}, k \in \mathbb{Z}\}$\\
              终边在$y$非正半轴
              & $\{\alpha \mid \alpha = k \cdot 360^{\circ} + 270^{\circ}, k \in \mathbb{Z}\}$\\
			  终边在$y=x$轴上              
              & $\{\alpha \mid \alpha = k \cdot 180^{\circ} + 45^{\circ}, k \in \mathbb{Z}\}$\\
              终边在坐标轴
              & $\{\alpha \mid \alpha = k \cdot 90^{\circ}, k \in \mathbb{Z}\}$\\
              第一象限角
              & $\{\alpha \mid \alpha = 0^{\circ} + k \cdot 360^{\circ} < \alpha < 90^{\circ} + k \cdot 90^{\circ}, k \in \mathbb{Z}\}$\\
              第二象限角
              & $\{\alpha \mid \alpha = 90^{\circ} + k \cdot 360^{\circ} < \alpha < 180^{\circ} + k \cdot 90^{\circ}, k \in \mathbb{Z}\}$\\
              第三象限角
              & $\{\alpha \mid \alpha = 180^{\circ} + k \cdot 360^{\circ} < \alpha < 270^{\circ} + k \cdot 90^{\circ}, k \in \mathbb{Z}\}$\\
              第四象限角
              & $\{\alpha \mid \alpha = 270^{\circ} + k \cdot 360^{\circ} < \alpha < 360^{\circ} + k \cdot 90^{\circ}, k \in \mathbb{Z}\}$\\
  \bottomrule
  \end{tabular}
\end{table}
\end{note}

\vspace{0.4cm}
\begin{remark}
因为角是由端点和端点出发的两条射线(始边和终边)构成的,因此在直角坐标系上轴线角要么是在$x$轴的非正半轴或非负半轴,要么就是在$y$轴的非正半轴或者非负半轴,而不是$x$的正半轴或负半轴(或$y$轴的正半轴或负半轴)。
\end{remark}

\vspace{0.4cm}

\begin{note}{已知角$\alpha$所在象限,求角$\displaystyle \theta = \frac{\alpha}{n} \mbox{或} \theta = n \alpha \left(n \in {\mathbb{N}}^\star \right) $ 所在象限}
\begin{enumerate}
\item 先将角$\alpha$的范围转化为含$k$的不等式;
\item 两边同乘或者同除$n$;
\item 对$k$进行奇偶性讨论,即
\begin{enumerate}
\item $k = 2n \left( n \in \mathbb{Z} \right)$
\item $k = 2n + 1 \left( n \in \mathbb{Z} \right)$
\end{enumerate} 
时不等式的结果,从而得到$\theta$所在的象限。
\end{enumerate}
\end{note}

\begin{exercise}
写出终边在$y$轴上的角的集合.
\end{exercise}

\begin{solution}
在$0^{\circ} \sim 360^{\circ}$范围内,终边在$y$轴上的角由两个:$90^{\circ} \mbox{和} 270^{\circ}$.\\
因此,所有与$90^{\circ}$
\end{solution}

\subsection{弧度制}

在初中我们知道“度”是将圆平均分成360份,期中的一份所对的角叫做$1$度

\begin{figure}
	\begin{center}
	\begin{tikzpicture}
	\def\R{5cm}
	\draw (0,0) circle[radius=\R];
	\draw (0,0) -- (0:\R) (0,0) -- (1:\R) (0,0) -- (57.30:\R);
	\end{tikzpicture}
	\end{center}
\end{figure}

\begin{definition}{角度制与弧度制}{degree-radian}
\begin{enumerate}
\item 角度:由角(圆心角)所对的\textcolor{third}{圆弧的长度}除以\textcolor{third}{圆的周长}再乘以\textcolor{third}{$360^{\circ}$}为角的角度,一般用“${}^{\circ} $”来标记,读作“度”。一度可以继续分为60“分”或3600“秒”,使用角度为单位度量角的单位制角角度制。

\item 弧度:圆心角所对的\textcolor{third}{圆弧的长度}除以\textcolor{third}{圆的半径}所得的值为角的弧度。一般用“$rad$”来标记弧度(通常可省略不写),读作“弧度”。用弧度作为单位度量角的单位制叫做\textcolor{third}{弧度制}。
\end{enumerate}
\end{definition}
%
%\begin{note}
%若半径为$r$的圆,圆心角$\theta$所对的圆弧长度为$l$,由初中知识我们可知圆的周长$C$为
%\begin{equation}
%C = 2 \pi r
%\end{equation}
%若圆周的360份中的一份所对应的圆心角$\phi$的角度是$1^{\circ}$,即$\displaystyle \abs{\phi}=\frac{1^{\circ}}{360^{\circ}}=\frac{1}{360}$,那么圆周360等分中的$l$份的圆心角$\theta$的角度若为$n^{\circ}$,那么$\displaystyle \theta = n \cdot \phi \mbox{且} \frac{1}{n} = \frac{l}{C}$:
%\begin{equation}
%\abs{\theta} = \frac{l}{C} \cdot 360^{\circ} = \frac{l}{2 \pi r} \cdot 360^{\circ}
%\end{equation}
%若半径为$r=1$的圆心角对应的弧长为$l=r$, 则$\displaystyle 1 \hspace{1mm} rad = \frac{\pi r}{l} \cdot 180^{\circ} $
%\end{note}

\begin{definition}{圆心角的弧度}{central-angle-radian}
在半径为$r$的$\odot O$中,弧长为$l$的弧所对的圆心角为$\alpha rad$则:
\begin{equation}
\displaystyle \abs{\alpha} = \frac{l}{r}
\end{equation}
\end{definition}

\begin{definition}{终边相同的角(弧度制)}{radian-set}
与角$\alpha$终边相同的角$\beta \in S$
\begin{equation}
S = \{\beta \mid \beta = \alpha + 2k\pi, k \in \mathbb{Z}\}
\end{equation}
\end{definition}

\begin{note}
由角度制和弧度制的定义可知
\begin{eqnarray}
180^{\circ} = \pi \hspace{1mm} rad \iff \left \{
\begin{array}{l}
\displaystyle 1^{\circ} = \frac{\pi}{180} \hspace{1mm} rad \approx 0.01745 \hspace{1mm} rad \vspace{0.5cm} \\
\displaystyle 1 \hspace{1mm} rad = {\left( \frac{180}{\pi} \right)}^{\circ} \approx 57.30^{\circ}
\end{array}	
\end{eqnarray}
\end{note}

\begin{definition}{扇形的弧长和面积公式}{arc-area-def}
若扇形的半径为$R$, 弧长为$l$, 面积为$S$, 且$\alpha \left(0 < \alpha < 2\pi \right)$为扇形的圆心角,则:
\begin{eqnarray}
l=\alpha R \hspace{5mm} \left(\mbox{弧长公式}\right) \\
\displaystyle S=\frac{1}{2}lR=\frac{1}{2}\alpha R^2 \hspace{5mm} \left(\mbox{面积公式}\right)
\end{eqnarray}
\end{definition}

\begin{note}
\begin{enumerate}
\item 零扇形的弧长和面积公式解题时,角的单位必须是弧度。
\item 求扇形面记得最大值时,利用面积公式转化为求二次函数最大值问题。
\item 在解决弧长和扇形面积问题时,要合理利用圆心角所在的三角形。
\end{enumerate}
\end{note}


\section{三角函数的概念}

\subsection{三角函数的概念}

给定任意角$\alpha \in \mathbb{R}$, 它的终边$OP$与单位圆交点$P$的坐标都是角$\alpha$的函数。

\begin{definition}{三角函数}{trig-function}
任意角$\alpha \in \mathbb{R}$, 点$P(x,y)$是角$\alpha$的终边$OP$与单位圆$\odot O$的交点,则:
\begin{enumerate}
\item 正弦函数就是点$P$的纵坐标$y$ ,记作$\sin{\alpha}$,即 $y=\sin{\alpha}$
\item 余弦函数就是点$P$的纵坐标$x$ ,记作$\cos{\alpha}$,即 $x=\cos{\alpha}$
\item 正切就是点$P$的纵坐标$y$与横坐标$x$的比值 ,记作$\tan{\alpha}$,即 $\displaystyle \frac{y}{x}=\tan{\alpha}, (x \neq 0)$
\end{enumerate}
\end{definition}

\begin{conclusion}
$\forall \alpha \in \mathbb{R}$,若$\alpha$终边上任意一点$P(x,y), (x, y) \neq (0, 0)$,则:
\begin{equation}
\begin{array}{l}
\displaystyle \sin{\alpha} = \frac{y}{\sqrt[]{x^2+y^2}} \\
\displaystyle \cos{\alpha} = \frac{x}{\sqrt[]{x^2+y^2}} \\
\displaystyle \tan{\alpha} = \frac{y}{x}
\end{array}
\end{equation}
\end{conslusion}
\subsection{同角三角函数的基本关系}

\begin{definition}{终边相同的角的同一三角函数}{trigo-equ1}
	\begin{equation}
	\begin{array}{l}
		\sin{\alpha} = \sin{\left(\alpha + k \cdot 2 \pi \right)}, k \in \mathbb{Z} \\
		\cos{\alpha} = \cos{\left(\alpha + k \cdot 2 \pi \right)}, k \in \mathbb{Z} \\
		\tan{\alpha} = \tan{\left(\alpha + k \cdot 2 \pi \right)}, k \in \mathbb{Z}	
	\end{array}
	\end{equation}
\end{definition}

利用\textcolor{third}{终边相同的角的同一三角函数值相等}结论,我们可以将求任意角的三角函数值转换到求$\displaystyle 0 \sim 2\pi$(即$0^{\circ} \mid 360^{\circ}$)之间角的三角函数值。

\begin{definition}{同角三角函数之间的关系}
	\begin{equation}
		{\sin{\alpha}}^2 + {\cos{\alpha}}^2 = 1\\
		
	\end{equation}
\end{definition}

\section{诱导公式}



\section{三角函数的图像和性质}

\begin{remark}
函数的三要素、三性质
\begin{enumerate}
\item 定义域
\item 对应更关系
\item 值域
\end{enumerate}
\begin{enumerate}
\item 单调性
\item 奇偶性
\item 周期性
\end{enumerate}
\end{remark}

\subsection{正弦函数、余弦函数的图像}
\subsection{正弦函数、余弦函数的性质}
\subsubsection{周期性}
\subsubsection{奇偶性}
\subsubsection{单调性}

\subsection{正切函数的图像}
\subsection{正切函数的性质}
\subsubsection{周期性}
\subsubsection{奇偶性}
\subsubsection{单调性}

\section{三角恒等变换}
\subsection{和角公式}
\subsection{差角公式}
\subsection{倍角公式}

\section{函数$y=A\sin{\omega x + \varphi}$}


\section{三角函数的应用}

\newpage
\begin{problemset}
\item 十九大指出中国的电动汽车革命早已展开,通过以新能源汽车代替汽/柴油车,中国正在大力实施一项将重塑全球汽车行业的计划,2020年某企业计划引进新能源汽车生产设备,通过市场分析,全年需投入固定成本3000万元,每生产$x$(百辆)需要另投入成本$y$(万元),且
\begin{equation*}
y = \left \{
	\begin{array}{l}
		10x^2+100x, 0 < x <40\\
		\displaystyle 501x + \frac{10000}{x}-4500, x \geq 40
	\end{array}
\end{equation*}
由市场调研知,每辆车售价$5$万元,且全年内生产的车辆当你那能全部销售完。
\begin{enumerate}
	\item 求出2020年的利润$S$(万元)关于年产量$x$(百辆)的函数关系式;(利润=销售额-成本)
	\item 当2020年产量为多少百辆时,企业所获利润最大?并求出最大利润。
\end{enumerate}

\vspace{4mm}
\item 已知某工厂生产机器设备的年固定成本为200万元。没生产1台还需另投入20万元。设该公司一年内共生产该机器设备$x$台并全部销售完,每台机器设备销售的收入为$R(x)$万元,且
\begin{equation*}
	R(x) = \left \{
	\begin{array}{l}
		\displaystyle 40 + \frac{800}{x}, 0 < x \leq 30\\
		\displaystyle \vspace{2mm} \frac{280\sqrt{x}+1000}{x}, x > 30
	\end{array}
\end{equation*}
\begin{enumerate}
	\item 求年利润$y$(万元)关于年产量$x$(台)的函数解析式;
	\item 当年产量为多少台时,该工厂生产所获得的的年利润最大?并求出最大年利润。
\end{enumerate}

\vspace{4mm}
\item 湖南娄底某高科技企业决定开发生产一款大型电子设备。生产这种设备的年固定成本为500万元,每生产$x$台,需要零投入成本$h(x)$(万元),当年生产量小于60台时$h(x)=x^2+20x$(万元);当年产量不少于60台时$\displaystyle h(x)=102x+\frac{9800}{x}-2080$。若每台设备的售价为100万元,通过市场分析,假设该企业生产的电子设备能全部售完。
\begin{enumerate}
	\item 求年利润$L(x)$(万元)关于年产量$x$(台)的函数关系式;
	\item 年产量为多少台时,该企业在这一电子设备的生产中获利最大?
\end{enumerate}
\end{problemset}


\nocite{*} 
\bibliography{reference}
\appendix

\chapter{证明}
\label{ch:证明}


本附录包括了

\section{集合与常用逻辑用语}

\section{一元二次函数、方程和不等式}

\section{函数的概念与性质}

\section{指数函数与对数函数}

\section{三角函数}

\chapter{方法论}
\section{解题思路的理解和来源}
平时大家评价一个孩子“聪明”或者“不聪明”的依据是看这个孩子对某件事或很多事的反应以及有没有他子集的看法。比如一个“聪明”的孩子,往往反应快、思路清楚,有自己的逐渐。那么我们任务“反应快、思路清晰、有主见”是聪明的前提。学习成绩好的同学,反应快、思路清晰、有主见就是他们的必备条件。\\
那么解题也是如此,必须反应快、思路清晰、有主见。同一道题,不同的同学从不同的角度去理解,由不同的看法最终汇聚成正确的解题过程,这是解题的必然。无论是推导、还是硬性套用、凭借经验做题,都是思路的一种。有的同学由开始思路不清渐渐转变为清楚,有的同学根本没有思路,这就形成了做题上的差距。\\
\section{如何训练解题能力}
训练数学解题思想可以从必要性思维开始。必要性思维也是解答数学试题的万能法门,相对也是最直接、最快捷的答题思想。所谓的必要性思维就是通过所求结论或者某一限定条件寻求前提条件的思维。
\section{从问题入手找条件}
我们发现稍有难度的考题,出题者都设置了种种障碍。如果从已知条件出发,联系所学知识顺退下去也可能解决问题,但是往往应为已知条件和所学知识发生的可能联系太多,从一个个岔路试下去经常越做越复杂,难以得到答案。如果从问题入手,思考想要获得所求所证,必须要做什么,这样

\section{解题表}

\begin{table}[htbp]
  \caption{解题表\label{tab:color thm}}
  \centering
  \begin{tabular}{cc}
  \toprule
              
              & \\
  \midrule
              你必须弄清楚问题
              & \multirow{3}[1]{=}{未知是什么?已知是什么?条件是什么?满足条件是否可能?要确定未知,条件是否充分?或者它是否不充分?或者是多余的?或者是矛盾的?} \\*
              
               & \multirow{2}[1]{=}画图,引入适当的符号\\*
              
               & \multirow{2}[1]{=} 把条件的各个部分分开。你能否把他们写下来?\\*
  \bottomrule
  \end{tabular}
\end{table}

\subsection{弄清问题}
\textbf{你必须弄清楚问题}:


\end{document}
