\chapter{函数的概念与性质}
\label{ch:函数的概念与性质}

\begin{introduction}
\item 函数的定义~\ref{def:function}
\item 单调性~\ref{def:monotonicity}
\item 单调性推论~\ref{cor:monotonicity-cor}
\item 复合函数~\ref{def:composite-function}
\end{introduction}


\section{函数的概念及表示}

\begin{definition}{函数}{function}
设$A$, $B$是非空的实数集,如果对于集合$A$中的\textcolor{third}{任意一个数$x$}, 按照\textcolor{third}{某种确定的对应关系$f$}, 在集合$B$中\textcolor{third}{都有唯一确定的数$y$}和它对应,那么就成 $f: A \rightarrow B$ 为从集合 $A$ 到集合 $B$ 的一个 \textcolor{third}{函数}, 记作
\begin{equation}
y = f(x), x \in A
\end{equation}
其中, $x$叫做\textcolor{third}{自变量},自变量$x$的取值范围$A$叫做函数的\textcolor{third}{定义域};与自变量$x$的值对应的$y$\textcolor{third}{值}叫做\textcolor{third}{函数值},函数值的取值范围(集合)$\{f(x) \mid x \in A \}$ 叫做函数的\textcolor{third}{值域}。
\end{definition}

\begin{note}
函数的三要素
\begin{enumerate}
\item 定义域
\item 值域
\item 对应关系
\end{enumerate}
函数的值域是集合$B$的子集。
\end{note}

\subsection{复合函数}
基本的初等函数犹如构成物质的最基本的不可分割的微粒一样。中学阶段的初等函数都可以通过对基本初等函数进行四则运算(加减乘除)后获得,然后我们通过对基本初等函数和初等函数进行“复合”运算可以得到更多的函数。

\begin{definition}{基本初等函数}{}
\begin{enumerate}
\item 常数函数: $f(x) = c, (c为常数)$
\item 幂函数: $f(x) = x^a$
\item 指数函数: $f(x) = a^x, (a>0 \mbox{且} a \neq 1)$
\item 对数函数: $f(x) = \log_{a}{x}, (a>0 \mbox{且} a \neq 1)$
\item 三角函数:
\begin{enumerate}
\item 正弦函数 $f(x) = \sin{x}$
\item 余弦函数 $f(x) = \cos{x}$
\end{enumerate}
\end{enumerate}
\end{definition}

\begin{definition}{初等函数}{}
\begin{enumerate}
\item 一次函数: $f(x) = kx + b, (k \neq 0)$实际上就由下列函数的加减乘除运算得来的
\begin{enumerate}
\item 常数函数: $g(x) = k, (k \neq 0)$和$q(x) = b, (b \in \mathbb{R})$;
\item 幂函数:$p(x) = x$ (幂指数为1);
\end{enumerate} 
即$f(x) = g(x) \cdot p(x) + q(x)$;
\item 二次函数: $f(x) = ax^2 + bx + c, (a \neq 0)$ 则是多个幂函数和常数函数组成
\begin{enumerate}
\item 常数函数: $g(x) = a, (a \neq 0)$,$m(x) = b$ 以及$n(x) = c$;
\item 幂函数: $p(x) = x^2$ 和 $q(x)=x$
\end{enumerate}
即$f(x) = g(x) \cdot p(x) + m(x) \cdot q(x) + n(x)$;
\item 正切函数:$\displaystyle f(x) = \tan{x}, (x \neq \frac{\pi}{2} + k \cdot \pi, k \in \mathbb{Z})$则是由
\begin{enumerate}
\item 正弦函数: $p(x) = \sin(x)$;
\item 余弦函数: $q(x) = \cos{x}$
\end{enumerate} 
\vspace{2mm} 相除获得,即$\displaystyle f(x) = \frac{\sin{x}}{\cos{x}}$; 
\item \vspace{2mm} “双刀函数”:$\displaystyle f(x) = x - \frac{1}{x}$由两个幂函数相减而成;
\item \vspace{2mm} “Nike函数”: $\displaystyle f(x) = x + \frac{1}{x}$由两个幂函数相加而得。
\end{enumerate}
\end{definition}

如果$z$是关于$y$的一个函数,而且$y$是关于$x$的一个函数,那么$z$也是关于$x$的一个函数。其实复合函数就是函数套函数。

\begin{definition}{复合函数}{composite-function}
假设有函数$y, u$
\begin{eqnarray}
y=f(u), (\mbox{定义域:}u \in D_f, \mbox{值域:}y \in M_f) \\
u=g(x), (\mbox{定义域:}x \in D_g, \mbox{值域:}u \in M_g)
\end{eqnarray}
若$M_g \cap D_f \neq \varnothing$,则$\forall x \in \left( M_g \cap D_f \right)$,都有唯一确定的$y$与之对应,则变量$x$与$y$之间通过变量$u$构成一种函数关系$y = f(u) = f(g(x))$,我们称函数$f(g(x))$为\textcolor{third}{复合函数},记作:
\begin{equation}
\begin{array}{l}
y = f \circ g(x) = f(g(x)): D_g \to M_f
\end{array}
\end{equation}
其中$x$叫复合函数的\textcolor{third}{自变量},自变量$x$的取值范围$D=\{ x \mid x \in D_g, \mbox{且} g(x) \in D_f \}$叫复合函数$f(g(x))$的定义域, $u$叫复合函数的\textcolor{third}{中间变量}, $y$叫复合函数的\textcolor{third}{因变量}
\end{definition}

\begin{figure}[htbp]
 \centering
 \begin{tikzpicture}[ele/.style={fill=black,circle,minimum width=.8pt,inner sep=1pt},every fit/.style={ellipse,draw,inner sep=-2pt}]
 
  \node[ele,label=left:$a$] (a1) at (0,11) {};    
  \node[ele,label=left:$b$] (a2) at (0,10) {};    
  \node[ele,label=left:$c$] (a3) at (0,9) {};
  \node[ele,label=left:$d$] (a4) at (0,8) {};

  \node["90:$g${:}$A \to B$"] (g1) at (3,6) {};
  \node["90:$f${:}$B \to C$"] (f1) at (7,6) {};
  \node["90:$f \circ g${:}$A \to C$"] (f1) at (5,5) {};
  
  \node[ele,label=left:$a$] (m1) at (0,4) {};    
  \node[ele,label=left:$b$] (m2) at (0,3) {};    
  \node[ele,label=left:$c$] (m3) at (0,2) {};
  \node[ele,label=left:$d$] (m4) at (0,1) {};
    
  \node[ele,,label=90:$1$] (b1) at (5,11) {};
  \node[ele,,label=90:$2$] (b2) at (5,10) {};
  \node[ele,,label=90:$3$] (b3) at (5,9) {};
  \node[ele,,label=90:$4$] (b4) at (5,8) {};

  \node[ele,,label=right:$1$] (n1) at (9,4) {};
  \node[ele,,label=right:$9$] (n2) at (9,3) {};
  \node[ele,,label=right:$4$] (n3) at (9,2) {};
  \node[ele,,label=right:$16$] (n4) at (9,1) {};
  
  \node[ele,,label=right:$1$] (c1) at (9,11) {};
  \node[ele,,label=right:$9$] (c2) at (9,10) {};
  \node[ele,,label=right:$4$] (c3) at (9,9) {};
  \node[ele,,label=right:$16$] (c4) at (9,8) {};

  \node[draw,label=90:$A$, fit= (a1) (a2) (a3) (a4),minimum width=2cm] {} ;
  \node[draw,label=90:$B$, fit= (b1) (b2) (b3) (b4),minimum width=2cm] {} ; 
  \node[draw,label=90:$C$, fit= (c1) (c2) (c3) (c4),minimum width=2cm] {} ; 
   
  \node[draw,label=90:$A$, fit= (m1) (m2) (m3) (m4),minimum width=2cm] {} ;
  \node[draw,label=90:$C$, fit= (n1) (n2) (n3) (n4),minimum width=2cm] {} ; 

  \draw[->,thick,shorten <=2pt,shorten >=2pt] (a1) -- (b4) -- (c4);
  \draw[->,thick,shorten <=2pt,shorten >=2] (a2) -- (b2) -- (c3);
  \draw[->,thick,shorten <=2pt,shorten >=2] (a3) -- (b1) -- (c1);
  \draw[->,thick,shorten <=2pt,shorten >=2] (a4) -- (b3) -- (c2);
  
  \draw[->,thick,shorten <=2pt,shorten >=2pt] (m1) -- (n4);
  \draw[->,thick,shorten <=2pt,shorten >=2] (m2) -- (n3);
  \draw[->,thick,shorten <=2pt,shorten >=2] (m3) -- (n1);
  \draw[->,thick,shorten <=2pt,shorten >=2] (m4) -- (n2);
 \end{tikzpicture}
\end{figure}

\begin{note}
    \begin{enumerate}
        \item 复合函数$f(g(x))$中我们也可以称$y=f(u)$为外函数,$u=g(x)$为内函数,且内函数的值域必须是外函数的定义域的子集。
        \item 给出$y=\mbox{解析式}$形式的复合函数时可以根据\textcolor{third}{基本初等函数}(5个)和\textcolor{third}{初等函数}(3个)“外型”,将各类函数中自变量$x$所在位置看作一个占位符号(例如$\sqcup$)进行合理拆分:
        \begin{enumerate}
            \item 基础初等函数:
            \begin{enumerate}
                \item 幂函数: $f(\sqcup) = {\sqcup}^a$
                \item 指数函数: $f(\sqcup) = a^{\sqcup}, (a>0 \mbox{且} a \neq 1)$
                \item 对数函数: $f(\sqcup) = \log_{a}{\sqcup}, (a>0 \mbox{且} a \neq 1)$
                \item 三角函数:
                \begin{enumerate}
                    \item 正弦函数 $f(\sqcup) = \sin{\sqcup}$
                    \item 余弦函数 $f(\sqcup) = \cos{\sqcup}$
                \end{enumerate}
            \end{enumerate}
            \item 初等函数:
            \begin{enumerate}
                \item 一次函数: $f(\sqcup) = k{\sqcup} + b, (k \neq 0)$
                \item \vspace{2mm} 二次函数: $f(\sqcup) = a{\sqcup}^2 + b{\sqcup} + c, (a \neq 0)$ 
                \item \vspace{2mm} 正切函数:$\displaystyle f(\sqcup) = \tan{\sqcup}, ({\sqcup} \neq \frac{\pi}{2} + k \cdot \pi, k \in \mathbb{Z})$
            \end{enumerate}
        \end{enumerate}
    \end{enumerate}
\end{note}

\vspace{3mm}
\begin{note}
\begin{enumerate}
\item 对于复合函数$f(g(x))$,其定义域仍然是求自变量$x$的取值范围,而不是$g(x)$的范围。
\item 对于相同对应关系下的函数$f(x), f(g(x)), \mbox{以及} f(h(x))$, 其中对应的$x, g(x) \mbox{以及} h(x)$的范围相同。
\end{enumerate}
\end{note}

\vspace{3mm}
\begin{note}
求定义域
\begin{enumerate}
\item 已知函数$f(x)$的定义域为$I$,求$f(g(x))$的定义域:实质为$g(x) \in I$, 根据$g(x)$的解析是求的出$x$的范围,即为函数$f(g(x))$的定义域;
\item 已知函数$f(g(x))$的定义域为$I$,求$f(x)$的定义域:实质为$x \in I$, 根据$g(x)$的解析是求的出$g(x)$的范围,即为函数$f(x)$的定义域;
\item 已知函数$f(g(x))$的定义域为$I$,求$f(h(x))$的定义域:实质为$x \in I$, 先根据$g(x)$的解析是求出$g(x)$的范围(即为函数$f(x)$的定义域),然后将其作为$h(x)$的范围,再根据$h(x)$的解析式求出$x$的范围。
\end{enumerate}
\end{note}

\subsection{函数图像的变换}

通过一个函数图像进行适当的变换得到另一个与之有关的函数的图像,我们称之为图像的变换。图像变换的实质其实就是图像\textcolor{third}{坐标的替换}

\begin{conclusion}{图像的对称变换}{}
\begin{enumerate}
\item \vspace{2mm} 函数$f(\alpha)$与函数$f(\beta)$关于直线$\displaystyle x=\frac{\alpha + \beta}{2}$对称:
\begin{enumerate}
\item \vspace{2mm} (\textbf{两个函数对称})函数 $f(x)$ 与 函数$f(2a-x)$ 的图象关于直线 $x=a$ 对称;其中 $\displaystyle \frac{x+ (2a -x)}{2}= a$;
\item \vspace{2mm} (\textbf{两个函数对称})函数 $f(a+x)$ 与 函数$f(a-x)$ 的图象关于直线 $x=a$对称($\displaystyle \frac{(a+x)+(a-x)}{2}=a$);
\item \vspace{2mm} (\textbf{两个函数对称})函数 $f(x-a)$ 与 函数$f(a-x)$ 的图象关于 $y$ 轴对称(关于$y$轴对称也即是关于直线$x=0$对称,其中$\displaystyle \frac{(x-a)+(a-x)}{2}= 0 )$;
\item \vspace{2mm} 函数 $f(x)$ 与 $f(-x)$ 的图象关于 $y$ 轴对称;
\end{enumerate}
\item \vspace{2mm} 函数 $f(x)$ 与 $-f(x)$ 的图象关于 $x$ 轴对称;
\item \vspace{2mm} 函数 $y = m + f(\alpha)$与 函数$y = n - f(\beta)$ 的图象关于点 $\displaystyle (\frac{\alpha + \beta}{2}, \frac{m+n}{2})$中心对称:
\begin{enumerate}
\item \vspace{2mm} (\textbf{两个函数对称})$0+f(x)$与$2b-f(2a-x)$关于点$\displaystyle (a=\frac{x+(2a-x)}{2}, b=\frac{0+2b}{2})$中心对称;
\item \vspace{2mm} (\textbf{一个函数对称}) 若函数$y=f(x)$的定义域为$I$, $\forall x \in I$ 满足 $f(a+x)=2b-f(a-x) \implies$函数$f(x)$的图像关于点$(a, b)$中心对称;
\item \vspace{2mm} (\textbf{两个函数对称})函数 $f(x)$ 与 $-f(-x)$ 的图象关于原点$\left(( 0, 0 \right)$中心对称;
\item \vspace{2mm} (\textbf{一个函数对称})函数$y=f(x)$的定义域为$I$, $\forall x \in I$, 都有$-x \in I$,且$f(x) = -f(-x)$(奇函数的定义),则函数$y=f(x)$的图象关于原点$\left(( 0, 0 \right)$中心对称;
\end{enumerate}
\item \vspace{2mm} 函数 $f(x)$ 与 $f^{-1}(x)$ ($f^{-1}(x)$是$f(x)$的反函数)的图象关于直线 $y=x$ 对称;
\end{enumerate}
\end{conclusion}

\vspace{3mm}
\begin{conclusion}{图像的平移变换}{}
\begin{enumerate}
\item 将函数 $f(x)$ 的图像沿 $x$ 轴平移 $\mid a \mid, a \in \mathbb{R}$ 个单位,得到函数$f(x+a)$的图像。$a>0$时向左移($x$轴的负半轴方向),$a<0$时向右移($x$轴的正半轴方向)。
\item 将函数 $f(x)$ 的图像沿 $y$ 轴平移 $\mid a \mid, a \in \mathbb{R}$ 个单位,得到函数$f(x) + a$的图像。$a>0$时向上平移($y$轴的正半轴方向),$a<0$时向下平移($y$轴的负半轴方向)。
\end{enumerate}
\textbf{记忆:} 
\begin{enumerate}
\item \textcolor{main}{自变量}坐标替换:“\textbf{图像左加右减}”
\item \textcolor{main}{因变量}坐标替换:“\textbf{图像上加下减}”
\end{enumerate}
\end{conclusion}

\vspace{3mm}
\begin{conclusion}{图像的伸缩变换}
\begin{enumerate}
\item 将 $f(x)$ 的图象上各点的纵坐标伸长( $a>1$ )或缩短( $0<a<1$ )到原来的 $a$ 倍,而横坐标不变,得到函数 $af(x), (a>0)$ 的图象;
\item 将 $f(x)$ 的图象上各点的横坐标伸长( $0<a<1$ )或缩短( $a>1$ )到原来的 $\displaystyle \frac{1}{a}$ 倍,而纵坐标不变,得到函数 $f(ax), (a>0)$ 的图象.
\end{enumerate}
\textbf{记忆:} 
\begin{enumerate}
\item \textcolor{main}{自变量}坐标替换:“\textbf{图像分伸整缩}”
\item \textcolor{main}{因变量}坐标替换:“\textbf{图像整升分降}”
\end{enumerate}
\end{conclusion}

\vspace{3mm}
\begin{conclusion}{图像的翻折变换}
\begin{enumerate}
\item 将函数 $f(x)$ 的图像中在 $x$ 轴下方的部分,沿$x$轴对折到$x$上方得到$\mid f(x) \mid$的图像。
\item 将函数 $f(x)$ 的图像中在 $x$ 轴正半轴的图像沿着$y$轴对折到$y$轴左侧,得到$f(\mid x \mid)$的图像。
\end{enumerate}
\textbf{记忆:} 
\begin{enumerate}
\item \textcolor{main}{自变量}坐标替换:“\textbf{图像右翻左}”,
\item \textcolor{main}{因变量}坐标替换:“\textbf{图像下翻上}”。
\end{enumerate} 
\end{conclusion}

\subsection{抽象函数}
\begin{table}[htbp]
  \caption{常用具体函数对应的抽象函数\label{tab:color thm}}
  \centering
  \begin{tabular}{cc}
  \toprule
              抽象函数的性质
              & 对应的具体函数模型\\
  \midrule
              $f(x) \pm f(y) = f(x \pm y)$ 
              & $f(x)=kx+b, (k \neq 0)$\\
       		  \midrule
              $f(x) \cdot f(y) = f(x \cdot y)$\\
              $\displaystyle \frac{f(x)}{f(y)} = f(\frac{x}{y})$
              & $f(x) = x^a $\\
              \midrule
              $f(x) \cdot f(y) = f(x + y)$\\
              $\displaystyle \frac{f(x)}{f(y)} = f(x - y)$
              & $f(x) = a^x, (a >0 \mbox{且} a \neq 1) $\\
              \midrule
              $f(x + y) = f(x) \cdot f(y)$\\
              $\displaystyle f(x - y) = \frac{f(x)}{f(y)}$
              & $f(x)=\log_{a}{x}, (a>0 \mbox{且} a \neq 1)$\\
              \midrule
              $f(x+y) = f(x)\cdot g(y) + g(x)\cdot f(y) $
              & $f(x) = \sin{x}, g(x)=\cos{x} $\\
              $\displaystyle f(x+y) = \frac{f(x)+f(y)}{1-f(x)f(y)}$
              & $f(x)=tanx$\\
  \bottomrule
  \end{tabular}
\end{table}

\section{函数的基本性质}

\subsection{函数的单调性与最值}

\begin{definition}{单调性定义}{monotonicity}
设函数 $f(x)$ 的定义域为 $I$, 区间 \underline{$D \subseteq I$}, 如果 \underline{$\forall x_1, x_2 \in D$}, 当 \underline{$x_1 < x_2$ }时, 都有\underline{$f(x_1)<f(x_2)$ }(或者 $f(x_1)>f(x_2)$),那么就称 $f(x)$ 在\underline{区间 $D$ }上\underline{\textbf{单调递增}} (或者\textbf{单调递减})\\
区间 $D$ 称为函数 f(x) 的\underline{\textbf{单调递增区间}} (或者 \textbf{单调递减区间}).
\end{definition}

\begin{corollary}{单调性推论}{monotonicity-cor}
设函数 $f(x)$ 的定义域为 $I$, 区间 \underline{$D \subseteq I$}, $\bigtriangleup x = x_1 - x_2, (x_1 \neq x_2), \bigtriangleup y = f(x_1) - f(x_2)$,则:
\begin{enumerate}
\item \vspace{2mm} 函数$f(x)$在区间$D$上\textcolor{third}{单调递增} $\iff \forall x_1, x_2 \in D, x_1 \neq x_2, \mbox{都有} \displaystyle \frac{\bigtriangleup y}{\bigtriangleup x} > 0$
\item \vspace{2mm} 函数$f(x)$在区间$D$上\textcolor{third}{单调递减} $\iff \forall x_1, x_2 \in D, x_1 \neq x_2, \mbox{都有} \displaystyle \frac{\bigtriangleup y}{\bigtriangleup x} < 0$
\end{enumerate}
\end{corollary}


\begin{table}[htbp]
  \caption{函数单调性运算\label{tab:color thm}}
  \centering
  \begin{tabular}{cccc}
  \toprule
              运算
              & 函数$f(x)$单调性
              & 函数$g(x)$单调性
              & 结果\\
  \midrule
              $+$ 
              & $\uparrow $
              & $\uparrow $
              & $\uparrow $\\
              $+$
              & $\downarrow $
              & $\downarrow $
              & $\downarrow $\\
              $-$
              & $\uparrow $
              & $\downarrow $
              & $\uparrow $\\
              $-$
              & $\downarrow $
              & $\uparrow $
              & $\downarrow $\\
              $ \times $
              & $\uparrow $
              & $\uparrow $
              & $\uparrow $\\
              $ \times $
              & $\downarrow $
              & $\downarrow $
              & $\downarrow $\\
  \bottomrule
  \end{tabular}
\end{table}



%\begin{note}{抽象函数的单调性}
%\end{note}


\begin{definition}{函数的最值}
设函数 $f(x)$ 的定义域为 $I$, 如果 $\exists M \in \mathbb{R}$ 满足:
\begin{enumerate}
\item $\forall x \in I $, 都有 $f(x) \leq M$  (或者 $f(x) \geq M$)
\item $\exists x_0 \in I$ , 使得 $f(x_0)=M$
\end{enumerate}
那么,我们称 $M$ 是函数 $y=f(x)$ 的 \textbf{最大值} (或者\textbf{最小值})
\end{definition}

\subsection{函数的奇偶性}

\begin{definition}{奇偶性定义}{parity}
设函数$f(x)$的定义域为$I$,如果$\forall x \in I$,都有 $-x \in I$,且$f(x)=f(-x)$(或$f(x)=-f(-x)$),则我们称函数$f(x)$为\textcolor{third}{偶函数}(或\textcolor{third}{奇函数})
\end{definition}

\subsection{函数的周期性}

\begin{definition}{周期性}{cyclicity}
\end{definition}


\section{幂函数}

\begin{definition}{幂函数}{power-function}
\textcolor{second}{形如}$f(x) = x^{\alpha}$这样的函数,我们称之为\textcolor{third}{幂函数}。其中$x$是自变量,$\alpha$是常数。
\end{definition}

\begin{note}
\begin{eqnarray*}
\mbox{幂函数} = \mbox{系数} \cdot \mbox{底数}^{\mbox{幂指数}}
\end{eqnarray*}
其中$\mbox{\textcolor{third}{系数}}=1$, \textcolor{third}{底数}为自变量,\textcolor{third}{幂指数}为常数。
\end{note}


\begin{figure}[htbp]
\centering
     \begin{tikzpicture}
      \begin{axis}[
      	  width=10cm,
      	  xmin=-10,xmax=10,
          ymin=-10,ymax=10,
          axis lines = center,
          legend pos = south east,
          xlabel = $x$,
          ylabel = {$y$},
          title={\mbox{常见的五个幂函数}}
      ]
      \addplot [
          domain=-7:7, 
          samples=100, 
          color=black,
      ]
      {x^(-1)};
      \addplot [
          domain=0:7, 
          samples=100, 
          color=green,
      ]
      {sqrt(x)};
      \addplot [
          domain=-7:7, 
          samples=100, 
          color=orange,
      ]
      {x};
      \addplot [
          domain=-5:5, 
          samples=100, 
          color=red,
      ]
      {x^2};
      \addplot [
          domain=-5:5, 
          samples=100, 
          color=blue,
      ]
      {x^3};
      \node[label={0:{(1,1)}},circle,fill,inner sep=1pt] at (axis cs:1,1) {};
      \node[pin=-45:{$f(x)=x^{-1}$}] at (axis cs:-0.6,-3){};
      \node[fill=white, text=green] at (axis cs:8,2.6){$f(x)=x^{\frac{1}{2}}$};
      \node[fill=white, text=orange] at (axis cs:-4.8,-5){$f(x)=x$};
      \node[fill=white, text=red] at (axis cs:-2.8,8){$f(x)=x^{2}$};
      \node[fill=white, text=blue] at (axis cs:-2.5,-8){$f(x)=x^{3}$};

      %\addlegendentry{$f(x)=x^{-1}$}     
      %\addlegendentry{$f(x)=x^{\frac{1}{2}}$}
      %\addlegendentry{$f(x)=x$} 
      %\addlegendentry{$f(x)=x^{2}$}
      %\addlegendentry{$f(x)=x^{3}$}
      \end{axis}
    \end{tikzpicture}
\end{figure}



\section{函数的应用(一)}